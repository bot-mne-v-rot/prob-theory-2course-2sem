\begin{titlepage}
	\centering
	
	{\Large Санкт-Петербургский Государственный Университет \par}
	
	\vspace{0.5cm}
	
	{\large Факультет математики и компьютерных наук\par}
	
	\vspace{4.5cm}
	
	{\Huge\bfseries Теория вероятностей\par}
	
	\vspace{1.5cm}
	
	{\LARGE\itshape Конспект основан на лекциях \\ Александра Игоревича Храброва\par}
	
	\vspace{1.5cm}
	
	{\large \today\par}
	\vfill
	
	\begin{minipage}{6in}
		\centering
		\raisebox{-0.5\height}{\includegraphics[width=0.46\textwidth]{spbgu}}
		\hspace*{.2in}
		\raisebox{-0.5\height}{\includegraphics[width=0.4\textwidth]{fmkn}}
	\end{minipage}


\end{titlepage}

\thispagestyle{empty}
\noindent Конспект основан на лекциях по математическому анализу, прочитанных Александром Игоревичем Храбровым студентам Факультета математики и компьютерных наук Санкт-Петербургского государственного университета в весеннем семестре 2019--2020 учебного года.

\vspace{0.5cm}
\noindent
В конспекте содержится материал 4-го семестра курса теории вероятностей.

\vspace{1.5cm}

\noindent
\begin{minipage}[t]{0.5\textwidth}
	\noindent 
	\textbf{\large Авторы:}
	
	\vspace{0.33cm}
	
	\noindent 
	\textit{Ольга Шиманская\\
	Андрей Кислицын}
	
    \vspace{1.5cm}
    
    \noindent \textbf{\large Консультант:}
	
	\vspace{0.33cm}
	
	\noindent \textit{Михаил Опанасенко}

\end{minipage}

\vfill

\noindent \copyright\ 2020 г.\\
Распространяется под лицензией Creative Commons Attribution 4.0 International License, см. \url{https://creativecommons.org/licenses/by/4.0/}.

\vfill

\noindent Последняя версия конспекта и исходный код: 

\begin{center}
	\url{https://www.overleaf.com/read/cnrvskqzwhcw}
\end{center}

\scriptsize\noindent Сайт СПБГУ: \url{https://spbu.ru}. \\
Сайт факультета МКН: \url{https://math-cs.spbu.ru}.

\normalsize