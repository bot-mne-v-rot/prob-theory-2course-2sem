\section{Случайные блуждания}

 \begin{theorem} \label{th:Zvoz}
     Случайное блуждание на $\mathbb{Z}$ возвратно тогда и только тогда, когда $p = \frac{1}{2}$ (то есть оно симметрично).
 \end{theorem}

 \begin{proof}
 Воспользуемся критерием возвратности.
     $$\sum\limits_{n=1} ^ {+\infty} p_{00}(n) = \sum\limits_{n=1} ^ {+\infty} p_{00}(2n) = \sum\limits_{n=1} ^ {+\infty} \binom{2n}{n}p^n (1 - p)^n.$$
    Знаем формулу(например, как следствие формулы Стирлинга)
     $$\binom{2n}{n}p^n(1-p)^n \sim \frac{(4p(1-p))^n}{\sqrt{\pi n}}.$$
    Тогда
     если $ p \neq\frac{1}{2}$ $4p(1-p) < 1$, то ряд сходящийся, и по критерию блуждание невозвратное.
Если $ p =\frac{1}{2}$ $\binom{2n}{n}p^n(1-p)^n \sim \frac{1}{\sqrt{\pi n}}$, то ряд расходится и блуждание возвратное.
 \end{proof}

 Теперь опишем произвольное симметричное случайное блуждание (то есть можем перейти не только на соседние клетки) на $\mathbb{Z}$ следующим образом.
 $\xi_1, \xi_2, \ldots$ --- независимые одинаково распределенные симметричные целочисленные случайные величины. Обозначим
 $S_n = \xi_1 + \ldots + \xi_n$.

 \begin{theorem}
     Если $\xi_k$ симметричные и имеют матожидание, то случайное блуждание возвратно.
 \end{theorem}

 

 Теперь рассмотрим случайные блуждания в $\mathbb{Z}^d$, вероятность перехода  в каждую сторону равна $\frac{1}{2d}$.

 \begin{theorem}[Пойя о возвращении]
     Такое случайное блуждание на $\mathbb{Z}^d$ возвратно если и только если $d = 1$ или $2$.
 \end{theorem}

 \begin{proof}
     Для $d = 1$ и $p = \frac{1}{2d} = \frac{1}{2}$ доказали в теореме \ref{th:Zvoz}. 

    Докажем для $d = 2$. Пусть $\vv{\xi_n}$ ~--- случайное блуждание вдоль в $\Z^2$, а $\eta_n$ и $\widetilde{\eta_n}$ --- блуждания вдоль прямых $y=x$ и $y = -x$, они независимы. Тогда
$$P(\vv{\xi_{2n}} = 0)= P(\eta_{2n} = 0, \widetilde{\eta_{2n}} = 0) = P(\eta_{2n} = 0)P( \widetilde{\eta_{2n}} = 0)=
         \left(\binom{2n}{n}\frac{1}{4^n}\right)^2 \sim \frac{1}{\pi n},$$ поэтому ряд $\sum P(\vv{\xi_{2n}} = 0)$ расходится и по критерию блуждание возвратно.

  

        Пусть $d = 3$.
        Тогда
        \begin{align*}
            p_{00}(2n) &= \sum\limits_{k + j \le n} \binom{2n}{k,k,j,j, n - j - k, n-j-k} \frac{1}{6^{2n}} \\&=
         \binom{2n}{n}\frac{1}{6^{2n}}\sum\limits_{k + j\le n}\left(\binom{n}{k,j,n-j-k}\right)^2 \le \left[\sum\limits_{k + j\le n}\binom{n}{k,j,n-j-k} = 3^n\right] \\&\le \binom{2n}{n}\frac{1}{6^{2n}}3^n\cdot\max\binom{n}{k,j,n-j-k} \sim \left[\max \sim 3^n \frac{3\sqrt{3}}{2\pi n}\right] \\&\sim \frac{4^n}{\sqrt{\pi n}}\frac{1}{4^n3^n3^n}3^n3^n\frac{3\sqrt{3}}{2\pi n} \\&= \frac{3\sqrt{3}}{2}\frac{1}{(\pi n)^{\frac{3}{2}}}
        \end{align*}
    Итого ряд сходящийся и возвратности нет.
     
    
     Пусть $d \ge 4$. Делаем проекцию c $\Z^d$ на $\mathbb{Z}^3$ и она будет невозвратной, но тогда и исходная очевидно тоже, иначе противоречие с невозвратностью для проекции.
 \end{proof}

 \begin{example}[задача о разорении]

   Два игрока, у которых $A$ и $B$ монет соответственно играют в орлянку, с вероятностью $q=1-p$ первый платит второму и с вероятностью $p$
     --- второй первому.
     Найдем вероятность разорения, обозначим за $\beta_k(x)$ вероятность на $k$-ом шаге
     оказаться в $B$, если на нулевом шаге мы находимся в точке $x$, $-A < x < B$.
     \begin{gather*}
         \beta_k(x) = p\beta_{k - 1}(x + 1) + q\beta_{k -1}(x-1);\\
         \beta_{k - 1}(x) \le \beta_k(x) \le 1.
     \end{gather*}
    Тогда существует 
    $\lim_{k\rightarrow\infty}\beta_k(x)$ обозначим его за $\beta(x) \le 1.$
     Получили
     \begin{gather*}
         \beta(x) = p\beta(x - 1) + q\beta(x +1);\\
         \beta(-A) = 0, \ \beta(B) = 1.
     \end{gather*}

     Нужно решить это соотношение.
     Если $p\neq q$, то $pt^2 - t + q = 0$, $t_1 = 1, t_2 = \frac{q}{p}$ ---  его корни.

     Тогда $at_1^n + bt_2^n$ ~--- решение соотношения, потому что 
     $$pt_1^n = pt_1^2\cdot t_1^{n - 2} =
         (t_1 - q)t_1^{n - 2} = t_1^{n - 1} - qt_1^{n - 2}.$$
    Таким образом,
     $\beta(x) = a + b(\frac{q}{p})^x$, осталось подобрать $a, b$ так, чтобы совпали значения в $-A, B$. Итого $$\beta(x) = \frac{(\frac{q}{p})^x - (\frac{q}{p})^{-A}}{(\frac{q}{p})^B - (\frac{q}{p})^{-A}}.$$
    
 \end{example}

