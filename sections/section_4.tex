\section{Случайные величины}

 \begin{definition}
     Пусть $(\Omega, \mathcal{F}, P) $ --- вероятностное пространство.
     \textit{Случайной величиной} называется измеримая функция $\xi\colon \Omega \rightarrow \mathbb{R}$.
 \end{definition}


 \begin{definition}\textit{Распределение случайной величины} $P_\xi(A)$ --- это мера на борелевских подмножествах, определённая следующим образом:
     $$P_\xi(A) = P(\xi \in A) =  P(\{\omega\in\Om\mid \xi(\omega) \in A\}).$$
 \end{definition}

Докажем корректность определения. Знаем, достаточно определить $P$ на ячейках: 
$$P_\xi(a, b] = P(a < \xi \le b) =
         P(\xi \le b) - P(\xi\le a).$$
Такие слагаемые определены, так как $\xi$ --- измеримая. Очевидно, эти функции однозначно задают распределение. Определим их.

\begin{definition} \textit{Функцией распределения} случайной величины называется 
     $$F_\xi(x) = P(\xi\le x).$$
 \end{definition}
 
 \begin{definition}
     $\xi, \eta$ имеют \textit{одинаковое распределение}, если $P_\xi = P_\eta$.
 \end{definition}

Отметим, что это равносильно равенству $P(\xi \le b) = P(\eta \le b)$  для всех $b$.

 \begin{properties}[функции распределения]
 \enewline
     \begin{enumerate}
         \item $F_\xi$ нестрого монотонно возрастает;
         \item $0\le F_\xi \le 1$;
         \item ${\lim}_{n\rightarrow -\infty} F_\xi (x) = 0$ и ${\lim}_{n\rightarrow +\infty} F_\xi (x) = 1$;
         \item $F_\xi$ непрерывна справа;
         \item $P(\xi < x) = {\lim}_{y\rightarrow x-}  F_\xi(y)$;
         \item $F_\xi$ непрерывна в точке $x_0$ равносильно тому, что $P(\xi = x_0) = 0$;
         \item $F_{\xi + c} (x) = F_\xi (x - c)$;
         \item $F_{c\xi} (x) = F_\xi(\frac{x}{c})$ при  $c > 0$; \qedhere
     \end{enumerate}
 \end{properties}
 
 \begin{proof}
 \enewline
    \begin{enumerate}
    \item[3.] Пусть $x_n\searrow -\infty$. Тогда множества $\{\xi \le x_n\}$ вложены, значит можем применить следующее свойство меры:
    $$\lim_{n\rightarrow -\infty} F_\xi (x) = \lim_{n \to \infty} F_{\xi}(x_n) = \underset{n\rightarrow \infty}{\lim} P(\xi\le x_n) = P \left(\overset{\infty}{\underset{n = 1}{\bigcap}} \{\xi\le x_n\}\right) = P(\varnothing) = 0.$$
    \item[4.]Проверим непрерывность в точке $x_0$, пусть $x_n\searrow x_0$.
                   Тогда 
                   $$\lim_{x \to \ x_0} F_\xi(x) = \underset{x\rightarrow\infty}{\lim} F_\xi (x_n)= \underset{x\rightarrow\infty}{\lim} P(\xi \le x_n) =
                       P\left(\overset{\infty}{\underset{n = 1}{\bigcap}} \{ \xi\le x_n\}\right) = P(\xi\le x_0).$$
    \item[5.] Пусть $x_n \nearrow x$, тогда
    $$\lim_{y\rightarrow x-} F_\xi(y)= \lim_{n \to \infty} F_\xi(x_n) = \lim P(\xi\le x_n) = P\left(\overset{\infty}{\underset{n = 1}{\bigcup}} \{\xi\le x_n\}\right) = P(\xi < x).$$ \qedhere
    \end{enumerate}
 \end{proof}
    
    Также заметим, что если функция $F$ обладает свойствами 1--4 то это функция распределения для некоторой случайной величины, так как можем определить вероятностную меру как $\mu((a, b]) = F(b) - F(a)$(знаем, что это мера, из теории меры).


 \begin{definition}
    Случайная величина $\xi$ --- \textit{дискретная}(или говорят, что она имеет \textit{дискретное распределение}), если
     $$\xi \colon \Omega \rightarrow Y,$$
     где множество $Y$ не более, чем счётно.
 \end{definition}
В дискретном вероятностном пространстве распределение устроено следующим образом:
  $$P_\xi(A) = \underset{k: y_k\in A}{\sum} P(\xi = y_k).$$
  Таким образом, распределение полностью определяется вероятностями $P(\xi = y_k)$.

 \begin{definition}Случайная величина имеет \textit{непрерывное распределение}, если её функция распределения непрерывна.
 \end{definition}
Как мы уже знаем, это равносильно тому, что  $\forall x\in\mathbb{R}$  $P(\xi = x) = 0$.
 \begin{definition}Случайная величина $\xi$ имеет \textit{абсолютно непрерывное распределение}, если у её функции распределения есть плотность относительно меры Лебега, то есть $p_\xi (t) \ge 0$ измеримая по Лебегу такая, что
     $$P_\xi(A) = \int_A p_\xi(t)\dd t.$$
 \end{definition}
 
Понятно, что для функция распределения такой величины считается как 
     $$F_\xi (x) = \int_{-\infty}^x p_\xi (t) \dd t.$$

 \begin{properties}[плотности распределения]
 \enewline
     \begin{enumerate}
         \item $P(\xi\in A) = P_\xi(A) = \intt_A p_\xi (t) \dd t$;
         \item $F_\xi(x) = \intt_{-\infty}^x p_\xi(t) \dd t$;
         \item Если $t_0$ --- точка непрерывности $p_\xi$, то
               $p_\xi(t_0) = F_\xi'(t_0)$ (на самом деле равенство есть почти везде, так как монотонно возрастающая функция дифференцируема почти везде);
     \end{enumerate}
 \end{properties}

 \begin{examples}[различных распределений]
 \enewline
     \begin{enumerate}
         \item Распределение Бернулли ($\xi\sim \text{Bern}(p)$).
                \begin{gather*}
                    0\le p\le 1, \ \xi \colon \Om \to \{0, 1\},  \\
                    P(\xi = 0) = 1 - p,\ P(\xi = 1) = p.
                \end{gather*}

         \item Биномиальное распределение ($\xi\sim \text{Binom}(p)$).
         \begin{gather*}
             \xi \colon \Omega \rightarrow \{0,1, \ldots, n\}, \\
             P(\xi = k) = \binom{n}{k}p^k(1-p)^{n - k}.
         \end{gather*}

         \item Распределение Пуассона ($\xi \sim \text{Pois}(\lambda)$).
         \begin{gather*}
             \lambda > 0, \ \xi \colon \Omega\rightarrow\{0, 1, 2, \ldots\},\\
             P(\xi = n) = \frac{\lambda^n e^{-\lambda}}{n!}.
         \end{gather*}
    

         \item Геометрическое распределение ($\xi\sim \text{Geom}(p)$).
         \begin{gather*}
             0< p < 1, \ \xi \colon \Omega\rightarrow \{1, 2, 3, \ldots\}, \\
             P(\xi = n) = p(1-p)^{n - 1}.
         \end{gather*}

         \item Дискретное равномерное распределение.
         \begin{gather*}
             \xi \colon \Omega\rightarrow \{a+1, \ldots, b\},\\
             P(\xi = n) = \frac{1}{b - a}.
         \end{gather*}
    
         \item Непрерывное равномерное распределение ($\xi \sim \mathcal{U}[a, b]$).
         \begin{gather*}
             \xi \colon \Omega\rightarrow [a, b],\\
             p_\xi (t) = \frac{1}{b - a}\mathbbm{1}_{[a, b]}(t).
         \end{gather*}

         \item Нормальное распределение ($\xi\sim \mathcal{N}(a, \sigma^2)$).
         \begin{gather*}
             a\in\mathbb{R}, \ \sigma > 0.\\
             p_\xi (t) = \frac{1}{\sqrt{2\pi}\sigma} e^{-\frac{(t-a)^2}{2\sigma^2}}.
         \end{gather*}

               Для $\xi\sim\mathcal{N}(0, 1)$ это называется стандартным нормальным распределением. Нетрудно заметить, что функция распределения для такой величины равна $\Phi(x)$.

         \item Экспоненциальное распределение ($\xi\sim \text{Exp}(\lambda)$).
         \begin{gather*}
             \lambda > 0, \ \xi \colon \Omega \rightarrow [0, +\infty), \\
             p_\xi (t) =\lambda e^{-\lambda t}\mathbbm{1}_{[0, +\infty)}(t).
         \end{gather*}
     \end{enumerate}
 \end{examples}

