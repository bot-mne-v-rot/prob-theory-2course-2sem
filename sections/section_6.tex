\section{Свертки мер}

    В этом паграфе $\mu$ и $\nu$ --- конечные меры на борелевских подмножествах $\mathbb{R}$.
    
     \begin{definition} \textit{Сверткой мер} $\mu$ и $\nu$ называется
         $$\mu * \nu(A) = \int\limits_\mathbb{R} \mu(A - x) \dd\nu(x).$$
     \end{definition}

     \begin{properties}[свёртки мер]
    \enewline
         \begin{enumerate}

             \item $\mu * \nu(A) = \intt_{\mathbb{R}^2}\mathbbm{1}_{A}(x+y) \dd\mu(x)\dd\nu(y)$;

             \item $\mu*\nu = \nu*\mu$;

             \item $(\mu_1*\mu_2)*\mu_3 (A)  = \intt_{\mathbb{R}^3}\mathbbm{1}(x_1+x_2+x_3)\dd\mu_1(x_1)\dd\mu_2(x_2)\dd\mu_3(x_3) = \mu_1*(\mu_2*\mu_3) (A)$;

             \item $\mu_1*\mu_2*\ldots*\mu_n (A)  = \intt_{\mathbb{R}^n}\mathbbm{1}(x_1+x_2+\ldots +x_n)\dd\mu_1(x_1)\dd\mu_2(x_2)\ldots \dd\mu_n(x_n)$;

             \item $(c\mu)*\nu = c\cdot\mu*\nu$;

             \item $(\mu_1+\mu_2)*\nu = \mu_1 * \nu + \mu_2 * \nu$;

             \item Пусть $\delta_a$ --- мера, такая что $\delta_a(\{a\}) = 1$ и $\delta_a(\{\mathbb{R}\setminus \{a\}\}) = 0$. Тогда
                   $\mu*\delta_0 = \mu$ (то есть $\delta_0$ --- единица с точки зрения свертки).

                   
         \end{enumerate}
     \end{properties}
     
      \begin{proof}
\enewline
         \begin{enumerate}
             \item По определению свёртки
             \begin{align*}
                 \mu * \nu(A) &= \int\limits_\mathbb{R} \mu(A - x) \dd\nu(x) \\&= \int\limits_{\mathbb{R}}\int\limits_\mathbb{R} \mathbbm{1}_{A - y}(x) \dd\mu(x)\dd\nu(y) \\&= \int\limits_{\mathbb{R}^2}\mathbbm{1}_A (x + y) \dd\mu(x)\dd\nu(y).
             \end{align*}
             \item[] Пункты 2-6 очевидно следуют из определения и пункта 1.
             \item[7.] По определению свёртки
             $$\mu*\delta_0 (A) = \int\limits_\mathbb{R} \mu(A - x)d\delta_0(x) =  \mu (A - 0) = \mu A.$$ \qedhere
         \end{enumerate}
                       
\end{proof}

     \begin{theorem}
         Пусть $p_\mu$ и $p_\nu$ --- плотности мер $\mu$ и $\nu$ относительно меры Лебега $\lambda$.
         Тогда $\mu*\nu$ имеет плотность
         $$p(t) = \int\limits_\mathbb{R} p_\mu (t - x) p_\nu (x) \dd x.$$
         Это называется \textit{свёрткой функцией}.
     \end{theorem}

     \begin{proof}
        Надо проверить, что 
        $$\mu*\nu(A) = \int\limits_A p(t)\dd t.$$ 
        По определению $p(t)$
        \begin{align*}
            \int\limits_A p(t)\dd t &= \int\limits_A\int\limits_\mathbb{R} p_\mu(t - x)p_\nu(x)\dd x\dd t \\&= \int\limits_{\mathbb{R}^2}\mathbbm{1}_A(t)p_\mu(t-x)p_\nu(x) \dd x\dd t \\&= \int\limits_{\mathbb{R}^2}\mathbbm{1}_A(x+y)p_\mu(y)p_\nu(x)\dd x\dd y \\&= \int\limits_{\mathbb{R}^2} \mathbbm{1}_A(x+y) \dd\mu(y)\dd\nu(x) = \mu*\nu(A).
        \end{align*}
       
     \end{proof}

     \begin{theorem}

        Пусть $\xi$ и $\eta$ независимые случайные величины. Тогда $P_{\xi + \eta} = P_\xi * P_\eta$.
     \end{theorem}

     \begin{proof}
        По определению распределения
        \begin{align*}
            P_{\xi + \eta}(A) &= P(\xi + \eta \in A) = P((\xi, \eta) \in B) = P_{(\xi, \eta)}(B) \\&=\int\limits_{\mathbb{R}^2}\mathbbm{1}_B (x, y) \dd P_{(\xi, \eta)}(x, y) = \int\limits_{\mathbb{R}^2}\mathbbm{1}_B(x, y)\dd P_{\xi}(x) \dd P_\eta(y) \\&=\int\limits_{\mathbb{R}^2}\mathbbm{1}_A(x + y)\dd P_{\xi}(x) \dd P_\eta(y) = P_\xi * P_\eta (A).
        \end{align*}
        
     \end{proof}

     \begin{examples}
\enewline
         \begin{enumerate}
             \item Свертка с дискретным распределением. Дискретное распределение можно описать как  $\nu = \sum p_x \delta_x$ (вес на нагрузку в точке).
            \begin{gather*}
                \mu*\nu = \sum p_x\mu*\delta_x;\\
                \mu*\delta_a(A) = \int\limits_\mathbb{R}\mu(A- x)d\delta_a(x) = \mu(A-a).\\
                \mu*\nu(A) = \sum p_x\mu*\delta_x(A) = \sum p_x\mu(A-x).
            \end{gather*}

             \item Если меры $\mu$ и $\nu$ с нагрузками в $\mathbb{N}\cup \{0\}$, то
    \begin{gather*}
        \nu = \sum q_n\delta_n, \ \mu = \sum p_n\delta_n;\\
        \mu*\nu(A) = \overset{\infty}{\underset{n = 0}{\sum}} q_n\mu(A-n), \quad A = \{k\}, \ k\in\mathbb{Z}; \\
        \mu*\nu(\{k\}) = \overset{k}{\underset{n = 0}{\sum}}q_np_{k - n}.
    \end{gather*}
         

             \item Пусть величины $\xi_1 \sim \text{Pois}(\lambda_1)$ и $\xi_2 \sim \text{Pois}(\lambda_2)$ независимые. Тогда
                   для $\xi_1$ веса равны $\frac{e^{-\lambda_1}\lambda_1^n}{n!}$, для $\xi_2$ веса равны $\frac{e^{-\lambda_2}\lambda_2^n}{n!}$.

                   Для $\xi_1+ \xi_2$ веса будут равны $$\overset{n}{\underset{k = 0}{\sum}}\frac{e^{-\lambda_1}\lambda_1^k}{k!}\frac{e^{-\lambda_2}\lambda_2^{n - k}}{(n - k)!} = \frac{e^{-\lambda_1 - \lambda_2}}{n!}\overset{n}{\underset{k = 0}{\sum}}\binom{n}{k} \lambda_1^k\lambda_2^{n - k} = \frac{e^{-(\lambda_1+\lambda_2)}}{n!}(\lambda_1+\lambda_2)^n.$$

                   Итого получили, что $\xi_1+\xi_2\sim \text{Pois}(\lambda_1+\lambda_2)$.
         \end{enumerate}
     \end{examples}
