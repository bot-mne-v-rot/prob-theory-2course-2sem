\section{Цепи Маркова}

 \begin{definition}
     Пусть $Y$ ~--- конечное или счетное множество (так называемое \textit{фазовое пространство}, \textit{пространство состояний}).
     $(\Omega, \mathcal{F}, P)$ ~--- вероятностное пространство, $\xi_0, \xi_1, \ldots \colon \Omega \Ra Y$ --- случайные величины.
Для любого $n$ выполнялось 
$$P(\xi_n = a_n | \xi_{n-1} = a_{n - 1}, \ldots, \xi_0 = a_0 ) = P(\xi_n = a_n | \xi_{n- 1} = a_{n - 1})$$
 при $P(\xi_{n - 1} = a_{n-1}, \ldots, \xi_0 = a_0) > 0$.
 Тогда последовательность случайных величин $\xi_0, \xi_1,  \ldots$ --- \textit{цепь Маркова}.
 \end{definition}
 \begin{examples}
\enewline
     \begin{enumerate}
         \item Случайное блуждание на $\mathbb{Z}$.
               Пусть $\eta_k$ ~--- независимые случайные величины, $\eta_k = 1$ с вероятностью $p$ и $-1$ с вероятностью $1-p$. $\xi_n = \eta_1 + \ldots + \eta_n$.
               Очевидно, что если мы стоим в какой-то позиции $\xi_{n - 1}$, то значение $\xi_n$ будет зависеть только от $\xi_{n - 1}$,
               поэтому это цепь Маркова. Итого $\xi_n = \xi_{n - 1} + \eta_n$

               Отметим, что не всякая последовательность реализуется (например, нельзя за четное число шагов попасть в нечетную позицию и наоборот).

         \item Есть прибор, у которого 2 состояния --- сломан и работает. Если он исправен, то через фиксированный квант времени с вероятностью $p$ он ломается, а с вероятностью $1-p$ остаётся исправным. Если же сломан, то с вероятностью $q$ он становится исправным и с вероятностью $1-q$ не меняет своего состояния.
               Это тоже цепь Маркова, так как состояние зависит только от предыдущего шага, а то, что было до этого, не важно.
     \end{enumerate}
 \end{examples}
 
  \begin{definition}
     Функция $\pi\colon Y \to [0, 1]$ ~--- распределение на $Y$, если  $\sum_{y\in Y} \pi(y) = 1$.
 \end{definition}

Заметим, что цепь Маркова определяется двумя величинами --- начальным распределением ($\pi_0 = P_{\xi_0}$ ~--- вероятность на $Y$) и функцией перехода ($p_n(a, b) = P(\xi_n = b | \xi_{n - 1} = a)$).


 \begin{definition}
     Назовем цепь Маркова \textit{однородной}, если $p_n(a, b) = p_{ab}$, то есть не зависит от $n$.
 \end{definition}

    Нетрудно заметить, что для случайного блуждания по $\mathbb{Z}$ цепь однородна, $p_{k, k + 1} = p$ и $p_{k, k - 1} = 1-p$.

 \begin{theorem}
     $$P(\xi_0 = a_0, \ldots, \xi_n = a_n) = \pi_0(a_0)\cdot p_{a_0, a_1}\cdot\ldots\cdot p_{a_{n - 1}, a_n}.$$ (эта последовательность называется \textit{траекторией}).
 \end{theorem}

 \begin{proof}
     Индукция по $n$.
     База $n=0$ --- определение.
     Переход $n - 1\Ra n$:
    \begin{align*}
        P(\xi_0 = a_0, \ldots, \xi_n = a_n) &= P(\xi_0 = a_0 , \ldots, \xi_{n - 1} = a_{n - 1})\cdot P(\xi_n = a_n | \xi_0 = a_0, \ldots, \xi_{n - 1} =
         a_{n - 1}) \\&= P(\xi_0 = a_0 , \ldots, \xi_{n - 1} = a_{n - 1})\cdot P(\xi_n = a_n | \xi_{n - 1} = a_{n - 1})  \\&=
         \pi_0(a_0)\cdot p_{a_0, a_1}\cdot\ldots\cdot p_{a_{n - 1}, a_n}.
    \end{align*}
 \end{proof}

 \begin{theorem}
     Если заданы $\pi_0\colon Y \Ra [0, 1]$ и $p\colon Y\times Y \Ra [0,1]$, такие что $\sum_{y\in Y}\pi_0(y) = 1$ и
     $\sum_{y\in Y} p_{xy} = 1$, то существует такое вероятностное пространство $(\Omega, \mathcal{F}, P)$ и цепь Маркова
     с начальным распределением $\pi_0$ и вероятностные переходы $p$.
 \end{theorem}

 
 \begin{theorem}
     Пусть $\pi_n = P_{\xi_n}$, то есть распределение после $n$ шагов. Его можно представить как вектор
     длины $\abs{Y}$. $P$ --- матрица переходов, то есть матрица $\abs{Y} \times \abs{Y}$, элемент с координатами $(a, b)$ которой равен $p_{ab}$. Тогда
     $\pi_n = \pi_0 P^n$.
 \end{theorem}

 \begin{proof}
     База $n=0$ очевидна, матрица в нулевой степени --- единичная, получаем $\pi_0 = \pi_0$.
     Переход $n - 1\Ra n $: надо проверить, что $\pi_n = \pi_{n - 1}P$.
    Рассмотри элемент этого вектора, соответствующий значению $a \in Y$:
     $$P(\xi_n = a) = \underset{y\in Y}{\sum} P(\xi_{n - 1} = y)P(\xi_n = a | \xi_{n - 1} = y) = \underset{y\in Y}{\sum } \pi_{n - 1}(y) p_{ya}.$$
     Последнее --- это произведение $\pi_{n - 1}$ и столбца $P$, соответствующего $a$.
 \end{proof}
 
 \begin{definition}
    Распределение $\pi$ называется \textit{стационарным}, если $\pi P = \pi$.
 \end{definition}
 
 Из предыдущей теоремы становится ясно, что стационарное распределение --- то, которое не меняется со временем.

 \begin{notation}
     Вероятность перехода из $a$ в $b$ за $n$ шагов 
    $$p_{ab}(n) = P(\xi_n = b | \xi_0 = a).$$
 \end{notation}


 \begin{example} Рассмотрим случайное блуждание на $\Z$.
     Выбираем сторону с вероятностью $\frac{1}{2}$. Пусть $\pi(y)$ --- стационарное распределение, тогда $$\frac{1}{2}\pi(y - 1) + \frac{1}{2}\pi(y+1) =
         \pi(y) \a \pi(y + 1) - \pi(y) = \pi(y) - \pi(y - 1)\Ra \pi(y + 1) - \pi (y) = \const. $$
     Эта константа не может не равняться нулю, так как иначе через некоторое количество шагов вероятность станет больше 1 или меньше 0, а такого быть не может.
     Значит, $\pi(y) = \pi(y + 1)$, то есть вероятность оказаться в точке для каждой точки одинакова, но такого тоже не бывает, поскольку мы знаем,
     что $\sum_{y\in Y} \pi(y) = 1$ и поэтому $\pi(y)$ не может не равняться нулю.
     Итого поняли, что случайное блуждание не имеет стационарного распределения.
 \end{example}

 \begin{theorem}[Маркова]
     Пусть пространство состояний $Y$ конечно и $p_{ab} > 0$ для всех $a, b\in Y$, тогда существует единственное $\pi$ --- стационарное распределение, причём
     $\pi(b) = \lim_{n\to+\infty}p_{ab}(n)$. Более того,
     существуют $c > 0$ и $\lambda \in (0, 1)$ такие,
     что 
     $$|\pi(b) - p_{ab}(n)|\le c\lambda^n\quad \forall a, b\in Y.$$ 
 \end{theorem}

Стоит отметить, что условие не зависит ни от начального распределения, ни начальной позиции. 

\begin{proof}
    Вспомним про теорему Банаха о сжатии: если есть $(X, \rho)$ -- полное метрическое пространство и $T:X \to X$ -- сжимающее отображение с коэффициентом $\lambda \in (0, 1)$ (т.е. это такое отображение, что $\rho(T(x), T(y)) \le \lambda \cdot \rho(x, y)$), тогда существует единственная неподвижная точка (т.е. такой $x'$, что $T(x')=x'$). Сведем нашу теорему к этой.
    
    В качестве полного метрического пространства возьмем $\R^d$, где $d$-- количество элементов в $Y$, а в качетве нормы: $\parallel x \parallel  := |x_1| + \dots + |x_d|$. Рассмотрим $X = \{ x \in \R^d : \parallel x \parallel = 1, x_i \ge 0 \}$ -- это соответствует всем распределениям. В качестве отображения логично взять умножение на матрицу перехода $T(x) := xP$. Проверим, что оно сжимающее.
    
    Пусть $z := y - x$, тогда нам надо проверить, что: \begin{gather*}
        \parallel T(y) - T(x) \parallel \le \lambda \parallel y - x \parallel \\
        \parallel T(z) \parallel \le \lambda \parallel x \parallel
    \end{gather*}
    Важно, что у $z$ сумма координат 0.
    Пусть $\delta := \min\limits_{a, b, \in Y} p_{ab} > 0$.
    Оценим $\parallel T(z) \parallel$: \begin{align*}
         \parallel T(z) \parallel &= \sum_{k=1}^d |(T(z))_k| \\
         &= \sum_{k=1}^d |\sum_{j=1}^d z_jp_{jk}| (\text{вспомним что } z_1 + \dots + z_d = 0) \\
         &= \sum_{k=1}^d |\sum_{j=1}^d z_j(p_{jk} - \delta)|  \\
         &\le \sum_{k=1}^d \sum_{j=1}^d 
         |z_j|(p_{jk} - \delta) \\
         &= \sum_{j=1}^d |z_j| \sum_{k=1}^d 
         (p_{jk} - \delta) (\text{вспомним что } p_{1k} + \dots + p_{kk} = 1) \\
         &= (1-d\delta)\sum_{j=1}^d |z_j| = (1-d\delta)\parallel z \parallel
    \end{align*}
    Осталось сказать, что $(1-d\delta)$ и есть $\lambda$ из теоремы Банаха о сжатии. Скорость сходимости тоже следует из теоремы Банаха.
\end{proof}

 \begin{corollary}
     Пусть $Y$ --- конечное множество, и для некоторого $n$ выполняется
     $$p_{ab}(n) > 0 \quad \forall a, b\in Y.$$
     Тогда существует единственное стационарное распределение $\pi$, такое что
     $$\underset{m\to+\infty}{\lim} p_{ab}(m) = \pi(b)\quad \forall a, b\in Y.$$
 \end{corollary}

 \begin{definition}
     Состояние $b$ \textit{достижимо} из состояния $a$, если $p_{ab}(n) > 0 $ для некоторого $n$.

     Два состояния называются \textit{сообщающимися}, если $a$ достижимо из $b$, а $b$ достижимо из $a$.

     Состояние $a$ --- \textit{существенное}, если для любого $b$, достижимого из $a$, следует, что $a$ достижимо из $b$.
 \end{definition}

 \begin{exercise}
     Докажите, что в любой конечной цепи существует хотя бы одно существенное состояние.
 \end{exercise}

 Обозначим 
 $$f_a(n)= P(\xi_n = a, \xi_{n - 1}\neq a, \ldots, \xi_1\neq a| \xi_0 = a),$$ то есть вероятность того, что мы впервые вернёмся в $a$ за $n$ шагов.

 \begin{definition}
     Если $\sum_{n = 1}^{\infty}f_a(n) = 1$, то $a$ --- \textit{возвратное} состояние.

     Если $\sum_{n = 1}^{\infty}f_a(n) < 1$, то $a$ --- \textit{невозвратное} состояние.
 \end{definition}

 \begin{definition}
     Если $p_{aa}(n) \to 0$ при $n \to \infty$, то $a$ --- \textit{нулевое} состояние.
 \end{definition}

Введём обозначение
     $$F_a = \sum \limits_{n = 1}^{\infty}f_a(n).$$


 \begin{theorem}[критерий возвратности]
     Состояние $a$ ~--- возвратно тогда и только тогда, когда $P_a = +\infty$, где 
     $$P_a = \sum \limits_{n = 1}^{\infty}p_{aa}(n),$$ а если $a$ невозвратно, то 
     $$F_a = \frac{P_a}{1+P_a}.$$
 \end{theorem}

 \begin{proof}
     Считаем, что $f_a(0) = 0 $ и $p_{aa}(0) = 1$. Заведём производящая функции
     \begin{gather*}
         \mathcal{P}(z) = \sum \limits_{n = 0}^{\infty} p_{aa}(n)z^n; \\
     \mathcal{F}(z) =
         \sum \limits_{n = 0}^{\infty} f_a(n)z^n.
     \end{gather*}
     Тогда нетрудно понять, что $F_a = \mathcal{F}(1)$ и $P_a = \mathcal{P}(1) - 1$.Также поймём, что 
     $$p_{aa}(n) = \sum\limits_{k = 0}^{n} f_a(k)p_{aa}(n - k) .$$
    Тогда верно
    \begin{align*}
        \mathcal{P}(z) - 1 &= \sum\limits_{n = 1}^\infty p_{aa}(n)z^n = \sum\limits_{n = 1}^\infty\sum\limits_{k = 1}^n f_a(k)p_{aa}(n - k)z^n \\&=
         \sum\limits_{k = 0}^\infty f_a(k)z^k\sum\limits_{m = 0}^\infty p_{aa}(m)z^m = \mathcal{F}(z)\mathcal{P}(z).
    \end{align*}
     Таким образом,
     $$\mathcal{P}(z) = 1 + \mathcal{F}(z)\mathcal{P}(z),$$
     и, следовательно,
     \begin{equation*}\label{eq:kritvoz}\tag{$*$}
         \mathcal{F}(z) = \frac{\mathcal{P}(z) - 1}{\mathcal{P}(z)}.
     \end{equation*}
   

     Если ряд $\sum p_{aa}$ сходится, то по теореме Абеля
     $$\underset{\to 1-}{\lim} \mathcal{P}(z) =P_a.$$

     Если ряд $\sum p_{aa} = +\infty$, то 
     $$\underset{z\to 1-}{\lim} \mathcal{P}(z) =P_a = +\infty$$
     (упражнение). Таким образом, равенство есть всегда, и можем перейти к пределу в (\ref{eq:kritvoz}):
    $$F_a = \underset{z\to 1-}{\lim} \mathcal{F}(z) = \underset{z\to 1-}{\lim}\frac{\mathcal{P}(z) - 1}{\mathcal{P}(z)} =\frac{P_a}{P_a + 1} \begin{cases}
    < 1, &  P_a < +\infty;\\
    = 1, & \text{иначе}.
    \end{cases}$$

 \end{proof}

 \begin{corollary}
     Невозвратное состояние является нулевым.
 \end{corollary}

 \begin{proof}
     Если $a$ невозвратно, то ряд $\sum p_{aa}$ сходится, а значит
     $p_{aa}(h) \to 0$ при $h\to \infty$, то есть $a$ ~--- нулевое.
 \end{proof}

 \begin{theorem}[солидарности]
     Сообщающиеся состояния возвратны/невозвратны (нулевые/ненулевые) одновременно.
 \end{theorem}

 \begin{proof}
     Пусть $a$, $b$ --- сообщающиеся состояния, то есть по определению $p_{ab}(i) > 0$, $p_{ba}(j) > 0$ для некоторых $i, j$. Тогда
     $$p_{bb}(i + j + k) \ge p_{ba}(j)p_{aa}(k)p_{ab}(i).$$
     
     Если $b$ ~--- нулевое, то 
     $$p_{bb}(i + j + k)\underset{k\to +\infty}{\to} 0\Ra p_{aa}(k)\Ra 0,$$
     то есть $a$ --- тоже нулевое.
     
     Если $a$ ~--- возвратное, то 
     $$\sum\limits_{k=1} ^ {+\infty} p_{aa}(k) = + \infty \Ra  +\infty =
         \sum\limits_{k=1} ^ {+\infty} p_{ba}(j)p_{aa}(k)p_{ab}(i) \le \sum\limits_{k=1} ^ {+\infty} p_{bb}(i + j + k),$$
     то есть $b$ ~--- возвратное по критерию.
 \end{proof}

 \begin{example}[управление запасами]
     Максимальное количество товаров на складе равно $s$.
     Если на складе $\le s$, то заказываем до максимума.
     Cпрос в $n$-й момент времени $\eta_n$ ~--- независимо одинаково распределённые случайные величины. Тогда последовательность величин
     $$\xi_{n + 1} = \begin{cases} \xi_n -\eta_{n + 1}, \xi_n > s;
     \\ S - \eta_{n + 1}, \mbox{ иначе.} \end{cases}$$ будет цепью Маркова.
 \end{example}

