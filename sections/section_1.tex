\section{Базовые понятия}

 \begin{definition}
     $\Om = \{\om_1, \om_2, \ldots, \om_n\}$~--- \textit{множество (пространство) элементарных событий}, если
     \begin{enumerate}
         \item $\om_i$ -- равновозможны;
         \item $\om_i$ и $\om_j$ не реализуемы одновременно (несовместны);
         \item Какая-то $\om_i$ случается;
     \end{enumerate}
 \end{definition}

 \begin{examples}
    \enewline
     \begin{enumerate}
         \item Монетка~--- орёл или решка($1/0$);
         \item Игральный кубик;
         \item Колода карт;
     \end{enumerate}
 \end{examples}
Здесь и далее за $\#A$ обозначается мощность множества $A$.
 \begin{definition}
    \textit{Случайное событие} --- это некоторое $A \subset \Om$.
    \textit{Вероятность случайного события} --- это $P(A)= \frac{\#A}{\#\Om}$.
 \end{definition}

Для примеров пространств элементарных событий приведём примеры случайных событий: 
 \begin{examples}
 \enewline
     \begin{enumerate}
         \item Орёл/решка;
         \item Чётное число очков/число очков больше трёх;
         \item Пики/красные старше валета;
     \end{enumerate}
 \end{examples}

 \begin{properties}[вероятности]
     \begin{enumerate}
         \item $P(\emptyset) = 0, \; P(\Om) = 1$;
         \item Если $A \cap B = \emptyset$ (говорят, что они \textit{несовместны}), то $P(A \cup B) = P(A) + P(B)$;
         \item $P(A \cup B) = P(A) + P(B) - P(A \cap B)$;
         \item $P(A \cup B) \leq P(A) + P(B)$;
         \item \textit{Формула включений-исключений}:
         \begin{multline*}
              P(A_1\cup A_2 \cup \ldots \cup A_n) = \sum_i{P(A_i)} - \sum_{i<j}P(A_i\cap A_j) \\+ \sum_{i<j<k}P(A_i\cap A_j \cap A_k) - \ldots +(-1)^{n-1}P(A_1\cap\ldots\cap A_n).
         \end{multline*}
     
         \item $P(\ov{A}) = 1 - P(A) = P(\Om) - P(A)$, где $\ov{A} = \Om \setminus A$;
     \end{enumerate}
 \end{properties}

 \begin{proof}
 \begin{enumerate}
     \item[5.]Индукция. База --- пункт 3. Переход $n \to n+1$.
    Обозначим $B = A_1 \cup \ldots \cup A_n$. Тогда по 3-му пункту
     $$P(B\cup A_{n+1}) = P(B) +  P(A_{n+1}) - P(B\cap A_{n+1}).$$
    Заметим, что 
    $$B\cap A_{n+1} = \bigcup_{i=1}^n(A_i\cap A_{n+1}).$$
    Тогда по индукционному переходу
    $$P(B\cap A_{n+1}) = \sum_{i\leq n}P(A_i\cap A_{n+1}) - \sum_{i<j\leq n}P(A_i\cap A_j \cap A_{n+1}) + \ldots.$$
    \item[6.] Следствие из первого и третьего. \qedhere
 \end{enumerate}
 \end{proof}


 \begin{definition} \textit{Условная вероятность}. Пусть $A \neq \emptyset, \ P(A) > 0$. Тогда вероятность $B$ при условии $A$ --- это
     $$P(B|A) = \frac{\#(A\cap B)}{\#A} = \frac{\#(A\cap B)/\#\Om}{\#A/\#\Om} = \frac{P(A\cap B)}{P(A)}.$$
 \end{definition}

 \begin{properties}[условной вероятности]
 \enewline
     \begin{enumerate}
         \item $P(A|A) = 1$.
              Если $A \subset B$, то $P(B|A) = 1$.
         \item $P(\emptyset|A) = 0$;
         \item Если $B \cap C = \emptyset$, то
               $$P(B \cup C| A) = P(B|A)+P(C|A).$$
     \end{enumerate}
 \end{properties}


 \begin{proof} Докажем пункт 3.
 \begin{align*}
     P(B\cup C| A) &= \frac{P((B \cup C)\cap A)}{P(A)} = \frac{P((B \cap A)\cup(C \cap A))}{P(A)} \\&= \frac{P(A\cap B)+P(C\cap A)}{P(A)} = P(B|A) + P(C|A).
 \end{align*}
     
 \end{proof}

 \begin{theorem}[формула полной вероятности]
Пусть $\Om = \bigsqcup_{k=1}^m{A_k}$. Тогда
     $$P(B) = \sum_{k=1}^m P(B|A_k)\cdot P(A_k).$$
     В частности, если $0 < P(A) < 1$, то
     $$P(B) = P(B|A) \cdot P(A)+P(B|\ov{A}) \cdot P(\ov{A}).$$
 \end{theorem}

 \begin{proof}
     $$P(B) = \sum_{k=1}^m P(B \cap A_k) = \sum_{k=1}^m \frac{P(B\cap A_k)}{P(A_k)}\cdot P(A_k).$$

 \end{proof}

 \begin{example} Есть две урны. В первой 3 белых и 5 чёрных шаров, во второй 5 белых и 5 чёрных шаров. Кладём из первой во вторую два шара, и берём шар из второй. Какова вероятность, что он белый (обозначим за $B$)?
Обозначим за $A_i$ событие "взяли $i$ белых шаров из первой". Тогда
\begin{gather*}
    P(B) = P(B|A_0)P(A_0) + P(B|A_1)P(A_1)+P(B|A_2)P(A_2).\\
    P(B|A_0) = \frac{5}{12}.\\
    P(B|A_1) = \frac{1}{2}.\\
    P(B|A_2) = \frac{7}{12}.\\
    P(A_0) = \frac{C_5^2}{C_8^2} = \frac{5}{14}.\\
    P(A_1) = \frac{3\cdot 5}{C_8^2} = \frac{15}{28}.\\
    P(A_2) = \frac{C_3^2}{C_8^2} = \frac{3}{28}.\\
    P(B) = \frac{23}{48}.
\end{gather*}
 \end{example}

 \begin{theorem}[формула Байеса]
Если $P(A), P(B) > 0$, то
     $$P(A|B) = \frac{P(B|A)P(A)}{P(B)}.$$
 \end{theorem}

 \begin{proof}
     $$\frac{P(B|A)P(A)}{P(B)} = \frac{\frac{P(A\cap B)}{P(A)}\cdot P(A)}{P(B)} = P(A|B).$$
 \end{proof}

 \begin{theorem}[Байеса]
Пусть $\Om = \bigsqcup_{k=1}^m{A_k}$, $P(B) > 0$ и $P(A_k) > 0$. Тогда
     $$P(A_k|B) = \frac{P(B|A_k)P(A_k)}{\sum_{i=1}^m P(B|A_i)P(A_i)}.$$
 \end{theorem}

 \begin{example} Турнир по олимпийской системе (проигравший выбывает). 16 участников, из них 2 сестры, известно, что они сыграли друг с другом.
     Какова вероятность, что этот матч был финальным? Обозначим события $B$ -- сёстры сыграли между собой (их матч состоялся), $A_i$~--- матч мог состояться в $i$-м туре.
    \begin{gather*}
        P(A_4|B) = \frac{P(B|A_4)P(A_4)}{\sum P(B|A_i)P(A_i)}.\\
        P(A_1) = \frac{1}{15}.\\
        P(A_2) = \frac{2}{15}.\\
        P(A_3) = \frac{4}{15}.\\
        P(A_4) = \frac{8}{15}.\\
        P(B|A_1) = 1.\\
        P(B|A_2) = \frac{1}{4}.\\
        P(B|A_3) = \frac{1}{16}.\\
        P(B|A_4) = \frac{1}{64}.\\
        P(A_4|B)  = \frac{1}{15}.
    \end{gather*}
 \end{example}

 \begin{definition}
     Случайные события $A$ и $B$ \textit{независимы}, если $P(A \cap B) = P(A)\cdot P(B)$.
 \end{definition}
 Равносильное определение:
 $$P(A \cap B) = P(A) P(B) \iff P(B|A) = \frac{P(A\cap B)}{P(A)} = P(B).$$
 \begin{definition}
     $A_1, \ldots, A_m$ \textit{независимы в совокупности}, если
     $$\forall{i_1, \ldots, i_k}\ P(A_{i_1}\cap\ldots\cap A_{i_k})=P(A_{i_1})\cdot\ldots\cdot P(A_{i_k}).$$
 \end{definition}

 \begin{lemma}\label{ex:nez}
     $A_1, \ldots , A_m$ независимы в совокупности $\Ra$
     $P(B_1\cap\ldots \cap B_m)=P(B_1)\cdot\ldots \cdot P(B_m)$, где $B_i = A_i$ или $\overline{A_i}$
 \end{lemma}
 

 
Отметим, что независимость в совокупности неравносильна попарной независимости.
Приведём пример. Пространство~--- множество пар чисел при кидании двух кубиков.
Обозначим события $A$~--- чётное на первом, $B$~--- чётная на втором, $C$~--- чётная сумма.
\begin{gather*}
P(A) = P(B) = P(C) = \frac{1}{2}.\\
P(A\cap C) = P(B \cap C) = P(A \cap B) = \frac{1}{4}.\\
\end{gather*}
Значит, эти события попарно независимы. 
\begin{gather*}
P(A \cap B \cap C) = P(A \cap B) = \frac{1}{4}.\\
P(A)P(B)P(C) = \frac{1}{8},
\end{gather*}
то есть они не независимы в совокупности. 

 \begin{definition}
     $B$ \textit{не зависит от совокупности событий} $A_1, \ldots, A_m$, если
     $$\forall{i_1, \ldots, i_k}\ P(B|A_{i_1},\ldots, A_{i_k})=P(B) \iff P(B \cap A_{i_1} \cap \ldots \cap A_{i_k}) = P(B)\cdot P(A_{i_1} \cap \ldots \cap A_{i_k}).$$
 \end{definition}

 \begin{theorem}[Эрдёша-Мозера]
    В турнире участвует $n$ волейбольных команд. Играют каждая с каждой, без ничей. Пусть $k$~--- наибольшее число, для которого всегда найдутся такие команды $a_1,\ldots, a_k$, что $a_i$ выиграла у $a_j$, если $i < j$. Тогда $k \leq 1 + [2\log_2{n}]$.

 \end{theorem}
 \begin{proof}
     Турнир~--- полный орграф (стрелочки от победителей к проигравшим). Подходящая цепочка~--- полный ациклический подграф. Пусть событие
     $A(a_1, \ldots, a_k)$~--- подошёл набор $a_1, \ldots a_k$. Тогда
     $P(A) = 2^{- {\binom{k}{2}}}$.
     Способов выбрать набор (выбрать $k$ команд и порядок на них) ~--- ${\binom{n}{k}} \cdot k!$.
Вероятность того, что какой-то набор подойдёт не превосходит
     $$2^{-{\binom{k}{2}}}\cdot \binom{n}{k}\cdot k!.$$
     Докажем, что если $k > 1 + [2\log_2{n}]$, то это значение меньше единицы, то есть существует граф, на котором нет такого набора. 
     \begin{gather*}
         k > 1 + [2\log_2{n}] \Ra \log_2 n < \frac{k - 1}{2} \Ra n < 2^{\frac{k-1}{2}}.\\
     2^{-{\binom{k}{2}}}\cdot {\binom{n}{k}}\cdot k! = 2^{-\frac{k \cdot (k-1)}{2}}\cdot \frac{n!}{(n-k)!}< 2^{-\frac{k \cdot (k-1)}{2}}\cdot n^k < 1.
     \end{gather*}
 \end{proof}
