\section{Ветвящийся процесс} 

Модель следующая. В начальный момент времени есть одна одна частица. Далее на каждом шаге каждая частица с вероятностью $f_k$ делится на $k$ частиц, причём
 $\sum_{k=0} ^ {+\infty} f_k = 1$.
Обозначим за $\eta_i$ количество частиц на $i$-м шаге, а $\xi_j^(i)$ --- количество потомков $j$-ой частицы (на каждом шаге своя нумерация) на $i$-м шаге. Тогда
\begin{gather*}
    \eta_0 = 1, \\
    \eta_1 = \xi_1^{(1)}, \\
    \eta_2 = \xi_1^{(2)} + \ldots + \xi_{\eta_1}^{(2)}, \ldots,
\end{gather*}
 причем $P(\xi_j^{(n)} = k) = f_k$.

 Пусть $G_n(z)$~--- производящая функция для $\eta_n$,
 $G(z)$~--- производящая функция для $\xi_i^{(k)}$. Из написанного выше $G_1 = G$.
 $$G_n(z) = \EE z^{\eta_n} = \sum\limits_{k=0} ^ {+\infty}  P(\eta_{n - 1} = k)\EE z^{\xi_1^{(n)} + \ldots + \xi_k^{(n)}} = \sum\limits_{k=0} ^ {+\infty}
     P(\eta_{n - 1} = k)G^k(z) = G_{n - 1}(G(z)).$$
 Поэтому $G_n(z) = \underbrace{G(G(\ldots (G}_{n \text{ раз}}(z))))$.

 Таким образом, поняли как устроена производящая функция, теперь посчитаем матожидание числа частиц:
 $$\EE \eta_n = G_n^\prime(1) = G_{n - 1}^\prime(G(1))\cdot G^\prime(1) = G_{n - 1}^\prime(1)\cdot G^\prime(1) = [\text{ по индукции }] = (\EE \eta_1)^n.$$


 \begin{theorem}
     Вероятность вырождения процесса~--- наименьший неотрицательный корень уравнения $G(x) = x$.
 \end{theorem}

 Заметим, что
 \begin{gather*}
     G(x) = \sum\limits_{k=0} ^ {+\infty} f_kx^k; \\
     G^\prime(x) = \sum\limits_{k=1} ^ {+\infty} kf_kx^{k - 1}\ge 0 \fpri x\in [0, 1]; \\
     G^{\prime \prime}(x) = \sum\limits_{k=2} ^ {+\infty} k(k - 1)f_kx^{k - 2}\ge 0\fpri x\in [0, 1].
 \end{gather*}
     Таким образом, функция монотонна и выпукла.
 \begin{proof}
     Обозначим
     $A_n = \{\eta_n = 0\}$, очевидно $A_n \subset A_{n + 1}$. Также понятно, что
     $P(A_n) = G_n(0) \le 1$.
     Раз события вложены, то вероятности неубывают и эти вероятности само собой ограничены, а значит есть предел, обозначим его за
     $q = \lim P(A_n) = \lim G_n(0)$.

     C одной стороны, $$G_{n + 1}(0) \to q.$$
     С другой, как мы выяснили выше,
     $$G_{n + 1}(0)= G(G_n(0))\to G(q).$$ 
     Получается $q = G(q)$, то есть найденный предел --- корень
     уравнения $G(x) = x$.

     Осталось доказать, что $q$~--- наименьший корень. Пусть $y$~--- наименьший неотрицательный корень. Тогда докажем, что $G_n(0)\le y$ всегда, тогда в
     пределе получим, что $q \le y$, доказав то, что нужно.
По условию $0\le y$, значит
$$G(0)\le G(y) = y \Ra G_2(0) \le G(y) = y$$ и так до n, получаем $G_n(0) \le y$.
 \end{proof}


 \begin{theorem}
     Если $m = \EE \eta_1 = 1$, $0 < b= \DD\eta_1$, $q_n$~--- вероятность вырождения к $n$-му шагу, $\gamma_n = q_{n+ 1} - q_n$~--- вероятность
     вырождения точно на $n$-м шаге. Тогда 
     $$\gamma_n \sim \frac{2}{bn^2} \fand 1-q_n\sim \frac{2}{bn}.$$
 \end{theorem}

 \begin{proof}
     Заведем вспомогательную функцию $g(x) = 1- G(1-x)$. $g(0) = 0$ (так как $G^\prime(1) = 1$)
    \begin{gather*}
        g^\prime(x) = G^\prime(1-x); \\
        g'(0) = m = 1;\\
        g^{\prime \prime}(x) = -G^{\prime \prime}(1-x);\\
        g^{\prime \prime}(0) = -G^{\prime \prime}(1) = -b,
    \end{gather*}
 поскольку
        \begin{gather*}
            b = \DD\eta_1 = G^{\prime \prime}(1) + G^\prime(1) - (G^\prime(1))^2 = G^{\prime \prime}(1).\\
            g(x) = x-\frac{bx^2}{2} + o(x^2);\\
            p_n = 1-q_n;\\ 
            g(p_n) = p_{n+1};\\
            \gamma_n = q_{n + 1} -q_n = p_n - p_{n+1};\\
        \end{gather*}.
     Пусть $a_n= \frac{1}{p_n}$. Тогда
     $$a_{n+1} - a_n = \frac{1}{p_{n+1}} -\frac{1}{p_n} = \frac{p_n - p_{n+1}}{p_{n+1}p_n} = \frac{p_n - g(p_n)}{p_ng(p_n)} = \frac{\frac{b}{2}p_n^2 + o(p_n^2)}{p_n(p_n - \frac{bp_n^2}{2} + o(p_n^2))}\sim \frac{b}{2}.$$
    И тогда по теореме Штольца.
    \begin{gather*}
        a_n\sim\frac{bn}{2} \Ra p_n \sim \frac{2}{bn}.\\
        \gamma_n = p_n - p_{n + 1} = p_np_{n+1} (a_{n+1} - a_n )\sim p_n^2\frac{b}{2} \sim \left(\frac{2}{bn}\right)^2\frac{b}{2} = \frac{2}{bn^2}.
    \end{gather*}
 \end{proof}

