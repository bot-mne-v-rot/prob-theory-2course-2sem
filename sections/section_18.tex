\section{Процесс восстановления}
 \begin{definition}
 Пусть $\xi_1, \xi_2, \ldots$ --- независимые одинаково распределенные неотрицательные случайные величины --- времена работы приборов. Как только один прибор
 ломается, покупается новый. Обозначим $S_0 = 0$, $S_n = \xi_1 + \ldots + \xi_n$~--- моменты восстановления приборов.

Введём обозначения
 $\Phi_n(t) = P(S_n \le t)$ --- функция распределения, $\nu(t)$ --- количество задействованных приборов к моменту $t$.
 Нетрудно заметить, что $\nu(t) = n$, если $S_{n - 1}\le t < S_n$. И
 тогда $P(\nu(t) = n)  = P(S_{n - 1}\le t < S_n) = \Phi_{n - 1}(t) - \Phi_n(t)$
    
    Функция
     $\mathcal{N}(t) = \EE \nu(t)$ называется \textit{функция восстановления}.
 \end{definition}

 \begin{remark}
\enewline
     \begin{enumerate}
         \item По определению
         \begin{align*}
             \mathcal{N}(t) &= \sum\limits_{n=1} ^ {+\infty} nP(\nu(t) = n) = \sum\limits_{n=1} ^ {+\infty} n(\Phi_{n - 1}(t) - \Phi_n(t)) \\&=
                   \Phi_{0}(t) - \Phi_1(t) +2 \Phi_{1}(t) - 2\Phi_2(t) + \ldots = \sum\limits_{n=0} ^ {+\infty} \Phi_n(t).
         \end{align*}
        

         \item Если $\xi_k$ целозначные, то
         \begin{gather*}
             \Phi_n(t) = \sum\limits_{k \le t} P(S_n = k);\\
             \mathcal{N}(t) =\sum \limits_{n = 0}^{\infty} \Phi_n(t) = \sum \limits_{n = 0}^{\infty} \sum \limits_{k \le t} P(S_n = k) = 
               \sum \limits_{k \le t} \sum \limits_{n = 0}^{\infty}  P(S_n = k).
         \end{gather*}
         Обозначим
               $$q_k = \sum \limits_{n = 0}^{\infty}  P(S_n = k).$$
     \end{enumerate}
 \end{remark}

 \begin{example} Пусть
     $P(\xi_k = 1) = p>0$, $P(\xi_k = 0) =  q = 1-p < 1$. Тогда
     $$q_k = \sum\limits_{n=0} ^ {+\infty} P(S_n = k) = \sum\limits_{n=0} ^ {+\infty} \binom{n}{k} p^kq^{n - k} = \left(\frac{p}{q}\right)^k\sum\limits_{n=0} ^ {+\infty}
         \binom{n}{k}q^n \le \left(\frac{p}{q}\right)^k\sum\limits_{n=0} ^ {+\infty} n^kq^n.$$ Последний ряд сходится при $q \neq 1$
     (по признаку Даламбера), значит все $q_k$ конечны, а тогда $\mathcal{N}(t)$ конечно при всех $t$.
 \end{example}

 \begin{theorem}
     Если $P(\xi_1 = 0 ) < 1$, то функция восстановления $\mathcal{N}(t)$ конечна при всех $t$.
 \end{theorem}

 \begin{proof}
 \begin{gather*}
     \lim\limits_{m\rightarrow +\infty} P(\xi_1\le\tfrac{1}{m}) = P(\xi_1 = 0) < 1; \\
     \{0\} = \bigcap\limits_{m=1} ^ {+\infty} \left[0, \frac{1}{m}\right].
 \end{gather*}
     Выберем $m$, так что $q= P(\xi_1\le \frac{1}{m}) < 1$.
    \begin{gather*}
        \wt{\xi_k} = \begin{cases}
        0, & \xi_k \le \frac{1}{m};\\
        1, & \xi_k > \frac{1}{m}.
        \end{cases}\\
        P(\wt{\xi_k} = 0) = q;\\
        P(\wt{\xi_k} = 1) = p = 1-q.
    \end{gather*}
    \begin{align*}
        \wt{\xi_k}\le m\xi_k &\Ra \wt{S_n} \le mS_n \Ra \{S_n \le t\} \subset\{\wt{S_n}\le mt\}
        \\&\Ra \Phi_n(t) = P(S_n\le t) \le P(\wt{S_n} \le mt) = \wt{\Phi}_n(mt) \\&\Ra \mathcal{N}(t) =
         \sum\limits_{n=0} ^ {+\infty} \Phi_n(t) \le \sum\limits_{n=0} ^ {+\infty} \wt{\Phi}_n(mt) = \wt{\mathcal{N}}(nt)
    \end{align*}
    --- эта функция конечна по примеру выше.
 \end{proof}

 \begin{definition}
     Дискретная случайная величина имеет \textit{решетчатое распределение}, если существует $a\in\mathbb{R}$ и $h>0$, такие что $\xi(\Omega) \in a+h\mathbb{Z}$ с
     вероятностью 1. Максимальное такое $h$~--- \textit{шаг решетки}.
 \end{definition}


 \begin{theorem}[восстановления]
    Пусть $\mathcal{N}(t+s) - \mathcal{N}(t) \rightarrow \frac{s}{\EE \xi_1}$. Тогда
     если $\xi_1$ нерешетчатая, то это верно для всех $s\in\mathbb{R}$, а если решетчатая, то для $s$ вида $hm$, где $m\in\mathbb{Z}$.
 \end{theorem}

 \begin{corollary}
     $$\frac{\mathcal{N}(t)}{t}\rightarrow \frac{1}{\EE \xi_1}.$$
 \end{corollary}

