\section{Формула полной вероятности. Формула и теорема Байеса. Примеры}


\begin{theorem}[формула полной вероятности]
    Пусть $\Om = \bigsqcup_{k=1}^m{A_k}$. Тогда
         $$P(B) = \sum_{k=1}^m P(B|A_k)\cdot P(A_k).$$
         В частности, если $0 < P(A) < 1$, то
         $$P(B) = P(B|A) \cdot P(A)+P(B|\ov{A}) \cdot P(\ov{A}).$$
     \end{theorem}
    
     \begin{proof}
         $$P(B) = \sum_{k=1}^m P(B \cap A_k) = \sum_{k=1}^m \frac{P(B\cap A_k)}{P(A_k)}\cdot P(A_k).$$
    
     \end{proof}
    
     \begin{example} Есть две урны. В первой 3 белых и 5 чёрных шаров, во второй 5 белых и 5 чёрных шаров. Кладём из первой во вторую два шара, и берём шар из второй. Какова вероятность, что он белый (обозначим за $B$)?
    Обозначим за $A_i$ событие "взяли $i$ белых шаров из первой". Тогда
    \begin{gather*}
        P(B) = P(B|A_0)P(A_0) + P(B|A_1)P(A_1)+P(B|A_2)P(A_2).\\
        P(B|A_0) = \frac{5}{12}.\\
        P(B|A_1) = \frac{1}{2}.\\
        P(B|A_2) = \frac{7}{12}.\\
        P(A_0) = \frac{C_5^2}{C_8^2} = \frac{5}{14}.\\
        P(A_1) = \frac{3\cdot 5}{C_8^2} = \frac{15}{28}.\\
        P(A_2) = \frac{C_3^2}{C_8^2} = \frac{3}{28}.\\
        P(B) = \frac{23}{48}.
    \end{gather*}
     \end{example}
    
     \begin{theorem}[формула Байеса]
    Если $P(A), P(B) > 0$, то
         $$P(A|B) = \frac{P(B|A)P(A)}{P(B)}.$$
     \end{theorem}
    
     \begin{proof}
         $$\frac{P(B|A)P(A)}{P(B)} = \frac{\frac{P(A\cap B)}{P(A)}\cdot P(A)}{P(B)} = P(A|B).$$
     \end{proof}
    
     \begin{theorem}[Байеса]
    Пусть $\Om = \bigsqcup_{k=1}^m{A_k}$, $P(B) > 0$ и $P(A_k) > 0$. Тогда
         $$P(A_k|B) = \frac{P(B|A_k)P(A_k)}{\sum_{i=1}^m P(B|A_i)P(A_i)}.$$
     \end{theorem}
    
     \begin{example} Турнир по олимпийской системе (проигравший выбывает). 16 участников, из них 2 сестры, известно, что они сыграли друг с другом.
         Какова вероятность, что этот матч был финальным? Обозначим события $B$ -- сёстры сыграли между собой (их матч состоялся), $A_i$~--- матч мог состояться в $i$-м туре.
        \begin{gather*}
            P(A_4|B) = \frac{P(B|A_4)P(A_4)}{\sum P(B|A_i)P(A_i)}.\\
            P(A_1) = \frac{1}{15}.\\
            P(A_2) = \frac{2}{15}.\\
            P(A_3) = \frac{4}{15}.\\
            P(A_4) = \frac{8}{15}.\\
            P(B|A_1) = 1.\\
            P(B|A_2) = \frac{1}{4}.\\
            P(B|A_3) = \frac{1}{16}.\\
            P(B|A_4) = \frac{1}{64}.\\
            P(A_4|B)  = \frac{1}{15}.
        \end{gather*}
     \end{example}
    \newpage
