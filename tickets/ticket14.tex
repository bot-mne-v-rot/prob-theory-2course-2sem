\section{Совместные распределения. Совместное распределение независимых случайных величин}

\begin{definition}
    Пусть $A$ --- борелевское из $R^n$, 
     $\vv{\xi} = (\xi_1, \ldots, \xi_n) \colon \Omega \rightarrow \mathbb{R}^n$. Тогда \textit{совместным (многомерным) распределением} этих величин называется 
     $$P_{\vv{\xi}}(A) = P(\vv{\xi} \in A).$$
 \end{definition}

Как и в случае с одномерным, многомерное распределение определеяется значением на ячейках:
     $$P_{\vv{\xi}}(a, b] = P(\vv{\xi} \in (a, b]) = P(a_1 < \xi_1 \le b_1, \ldots, a_n < \xi_n \le b_n).$$


     Отметим, что $P_{\overset{\rightarrow}{\xi}}$ определяет $P_{\xi_k}$, но не наоборот:
     $$A\subset\mathbb{R}; \  P_{\xi_1}(A) = P(\xi_1\in A) = P(\vv{\xi}\in A\times\mathbb{R}^{n - 1}) = P_{\vv{\xi}} (A\times\mathbb{R}^{n - 1}).$$


В другую сторону: рассмотрим величины $\xi_1$ и $\xi_2$ такие, что $P(\xi_i = 0) = P(\xi_i = 1) = \frac{1}{2}$.
     Если $\xi_1 = \xi_2$, то у нас 2 события с вероятностью $\frac{1}{2}$ ---  $(0, 0)$ и $(1, 1)$.
     Если же они независимы (например, подбрасывание монеток), то 4 события с вероятностью $\frac{1}{4}$.
     То есть, получили разные совместные распределения.


 \begin{definition}
     Случайные величины $\xi_1, \ldots, \xi_n$ \textit{независимы}, если 
     для любых множеств $A_1, \ldots, A_n \subset \mathbb{R}$ случайные события
     $\{\xi_1\in A_1\}, \ldots, \{\xi_n \in A_n\}$ независимы.
 \end{definition}

\begin{remark}
    Если случайные величины $\xi_1, \ldots, \xi_n$ независимы, то
    \begin{gather*}
        P(\xi_1\in A_1, \ldots, \xi_n\in A_n) = P(\xi_1\in A_1)\cdot\ldots\cdot P(\xi_n\in A_n), \\
        P_{\vv{\xi}}(A_1\times\ldots\times A_n) = P_{\xi_1}(A_1)\cdot\ldots\cdot P_{\xi_n} (A_n).
    \end{gather*}
\end{remark}
    


 \begin{theorem}
     Независимость величин $\xi_1, \ldots, \xi_n$ равносильно тому, что $P_{\vv{\xi}}$ --- произведение мер $P_{\xi_1}, \ldots, P_{\xi_n}$.
 \end{theorem}

 \begin{proof}
     Достаточно доказать равенство на ячейках:
     $$P_{\vv{\xi}}(a, b] = P_{\xi_1}((a_1, b_1])\cdot\ldots\cdot P_{\xi_n}((a_n, b_n]).$$
     Из замечания выше видно, что оно есть.
 \end{proof}