\section{Случайная величина. Распределение случайной величины. Свойства функций распределения}

\begin{definition}
    Пусть $(\Omega, \mathcal{F}, P) $ --- вероятностное пространство.
    \textit{Случайной величиной} называется измеримая функция $\xi\colon \Omega \rightarrow \mathbb{R}$.
\end{definition}


\begin{definition}\textit{Распределение случайной величины} $P_\xi(A)$ --- это мера на борелевских подмножествах, определённая следующим образом:
    $$P_\xi(A) = P(\xi \in A) =  P(\{\omega\in\Om\mid \xi(\omega) \in A\}).$$
\end{definition}

Докажем корректность определения. Знаем, достаточно определить $P$ на ячейках: 
$$P_\xi(a, b] = P(a < \xi \le b) =
        P(\xi \le b) - P(\xi\le a).$$
Такие слагаемые определены, так как $\xi$ --- измеримая. Очевидно, эти функции однозначно задают распределение. Определим их.

\begin{definition} \textit{Функцией распределения} случайной величины называется 
    $$F_\xi(x) = P(\xi\le x).$$
\end{definition}

\begin{definition}
    $\xi, \eta$ имеют \textit{одинаковое распределение}, если $P_\xi = P_\eta$.
\end{definition}

Отметим, что это равносильно равенству $P(\xi \le b) = P(\eta \le b)$  для всех $b$.

\begin{properties}[функции распределения]
\enewline
    \begin{enumerate}
        \item $F_\xi$ нестрого монотонно возрастает;
        \item $0\le F_\xi \le 1$;
        \item ${\lim}_{n\rightarrow -\infty} F_\xi (x) = 0$ и ${\lim}_{n\rightarrow +\infty} F_\xi (x) = 1$;
        \item $F_\xi$ непрерывна справа;
        \item $P(\xi < x) = {\lim}_{y\rightarrow x-}  F_\xi(y)$;
        \item $F_\xi$ непрерывна в точке $x_0$ равносильно тому, что $P(\xi = x_0) = 0$;
        \item $F_{\xi + c} (x) = F_\xi (x - c)$;
        \item $F_{c\xi} (x) = F_\xi(\frac{x}{c})$ при  $c > 0$; \qedhere
    \end{enumerate}
\end{properties}

\begin{proof}
\enewline
   \begin{enumerate}
   \item[3.] Пусть $x_n\searrow -\infty$. Тогда множества $\{\xi \le x_n\}$ вложены, значит можем применить следующее свойство меры:
   $$\lim_{n\rightarrow -\infty} F_\xi (x) = \lim_{n \to \infty} F_{\xi}(x_n) = \underset{n\rightarrow \infty}{\lim} P(\xi\le x_n) = P \left(\overset{\infty}{\underset{n = 1}{\bigcap}} \{\xi\le x_n\}\right) = P(\varnothing) = 0.$$
   \item[4.]Проверим непрерывность в точке $x_0$, пусть $x_n\searrow x_0$.
                  Тогда 
                  $$\lim_{x \to \ x_0} F_\xi(x) = \underset{x\rightarrow\infty}{\lim} F_\xi (x_n)= \underset{x\rightarrow\infty}{\lim} P(\xi \le x_n) =
                      P\left(\overset{\infty}{\underset{n = 1}{\bigcap}} \{ \xi\le x_n\}\right) = P(\xi\le x_0).$$
   \item[5.] Пусть $x_n \nearrow x$, тогда
   $$\lim_{y\rightarrow x-} F_\xi(y)= \lim_{n \to \infty} F_\xi(x_n) = \lim P(\xi\le x_n) = P\left(\overset{\infty}{\underset{n = 1}{\bigcup}} \{\xi\le x_n\}\right) = P(\xi < x).$$ \qedhere
   \end{enumerate}
\end{proof}
   
   Также заметим, что если функция $F$ обладает свойствами 1--4 то это функция распределения для некоторой случайной величины, так как можем определить вероятностную меру как $\mu((a, b]) = F(b) - F(a)$(знаем, что это мера, из теории меры).

