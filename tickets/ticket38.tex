\section{Интегральная теорема Муавра–Лапласа. Теорема Пуассона}

\begin{corollary}[теорема Муавра-Лапласа] Пусть
    $\xi_1, \ldots$ --- независимые испытания Бернулли с вероятностью успеха $p\in (0, 1)$, $S_n = \underset{i = 1}{\overset{n}{\sum}} \xi_i$.
    Тогда $$P\left(\frac{S_n - np}{\sqrt{npq}}\le x\right) \rightrightarrows \Phi(x).$$
\end{corollary}

\begin{proof}
    $\EE \xi_1 = p$, $ \DD\xi_1 = pq$, подставим в теорему.
\end{proof}

\begin{theorem}[Пуассона]
    Пусть 
    \begin{gather*}
        P(\xi_{nk} = 1) = p_{nk}, \ P(\xi_{nk} = 0) = 1- p_{nk} = q_{nk};\\
        a_n = \underset{1\le k\le n}{\max} p_{nk}\underset{n\rightarrow\infty}{\rightarrow} 0, \ p_{n1} +\ldots + p_{nn}\rightarrow \lambda > 0;
    \end{gather*}
   события $\xi_{nk}$ независимы при фиксированном $n$.
    Тогда 
    $$P(S_n = m)\underset{n\rightarrow\infty}{\rightarrow} \frac{e^{-\lambda} \lambda^m}{m!},$$
    где $S_n = \xi_{n1} + \ldots + \xi_{nn}$.
\end{theorem}

\begin{proof} Докажем методом характеристических функций.
    $$\phi_{\xi_{nk}}(t) = \EE e^{it\xi_{nk}} = p_{nk}e^{it} + 1 - p_{nk}.$$
   Надо доказать, что характеристическая функция
    $\phi_{S_n}(t)$
    стремится к $e^{\lambda(e^{it} - 1)}$  (так называемая характеристическая функция Пуассона), и тогда равенство из теоремы очевидно верно.
     $$\phi_{S_n}(t) = \prod\limits_{k=1} ^ n \phi_{\xi_{nk}}(t) = \prod\limits_{k=1} ^ n (1+p_{nk}(e^{it} - 1)),$$
    Прологарифмируем, тогда надо показать, что
    $$\sum \ln(1+p_{nk}(e^{it} - 1)) \to \lambda (e^{it} - 1).$$
   Распишем левую часть:
    $$\sum \ln(1+p_{nk}(e^{it} - 1)) = \sum((p_{nk}(e^{it} - 1)) + O(p_{nk}^2)) \rightarrow  \lambda (e^{it} - 1) + \sum\limits_{k=1} ^ n O(p_{nk} ^ 2).$$
   Оценим второе слагаемое:
    $$\sum\limits_{k=1} ^ n O(p_{nk} ^ 2) \le \sum\limits_{k=1} ^ n O(a_np_{nk}) = a_n O\left(\sum p_{nk}\right) \le Ca_n \rightarrow 0.$$

\end{proof}