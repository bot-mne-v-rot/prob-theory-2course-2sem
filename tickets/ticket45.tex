\section{Марковские цепи. Примеры. Вероятность фиксированной траектории. Теорема существования (без доказательства)}

\begin{definition}
    Пусть $Y$ ~--- конечное или счетное множество (так называемое \textit{фазовое пространство}, \textit{пространство состояний}).
    $(\Omega, \mathcal{F}, P)$ ~--- вероятностное пространство, $\xi_0, \xi_1, \ldots \colon \Omega \Ra Y$ --- случайные величины.
Для любого $n$ выполнялось 
$$P(\xi_n = a_n | \xi_{n-1} = a_{n - 1}, \ldots, \xi_0 = a_0 ) = P(\xi_n = a_n | \xi_{n- 1} = a_{n - 1})$$
при $P(\xi_{n - 1} = a_{n-1}, \ldots, \xi_0 = a_0) > 0$.
Тогда последовательность случайных величин $\xi_0, \xi_1,  \ldots$ --- \textit{цепь Маркова}.
\end{definition}
\begin{examples}
\enewline
    \begin{enumerate}
        \item Случайное блуждание на $\mathbb{Z}$.
              Пусть $\eta_k$ ~--- независимые случайные величины, $\eta_k = 1$ с вероятностью $p$ и $-1$ с вероятностью $1-p$. $\xi_n = \eta_1 + \ldots + \eta_n$.
              Очевидно, что если мы стоим в какой-то позиции $\xi_{n - 1}$, то значение $\xi_n$ будет зависеть только от $\xi_{n - 1}$,
              поэтому это цепь Маркова. Итого $\xi_n = \xi_{n - 1} + \eta_n$

              Отметим, что не всякая последовательность реализуется (например, нельзя за четное число шагов попасть в нечетную позицию и наоборот).

        \item Есть прибор, у которого 2 состояния --- сломан и работает. Если он исправен, то через фиксированный квант времени с вероятностью $p$ он ломается, а с вероятностью $1-p$ остаётся исправным. Если же сломан, то с вероятностью $q$ он становится исправным и с вероятностью $1-q$ не меняет своего состояния.
              Это тоже цепь Маркова, так как состояние зависит только от предыдущего шага, а то, что было до этого, не важно.
    \end{enumerate}
\end{examples}

 \begin{definition}
    Функция $\pi\colon Y \to [0, 1]$ ~--- распределение на $Y$, если  $\sum_{y\in Y} \pi(y) = 1$.
\end{definition}

Заметим, что цепь Маркова определяется двумя величинами --- начальным распределением ($\pi_0 = P_{\xi_0}$ ~--- вероятность на $Y$) и функцией перехода ($p_n(a, b) = P(\xi_n = b | \xi_{n - 1} = a)$).


\begin{definition}
    Назовем цепь Маркова \textit{однородной}, если $p_n(a, b) = p_{ab}$, то есть не зависит от $n$.
\end{definition}

   Нетрудно заметить, что для случайного блуждания по $\mathbb{Z}$ цепь однородна, $p_{k, k + 1} = p$ и $p_{k, k - 1} = 1-p$.

\begin{theorem}
    $$P(\xi_0 = a_0, \ldots, \xi_n = a_n) = \pi_0(a_0)\cdot p_{a_0, a_1}\cdot\ldots\cdot p_{a_{n - 1}, a_n}.$$ (эта последовательность называется \textit{траекторией}).
\end{theorem}

\begin{proof}
    Индукция по $n$.
    База $n=0$ --- определение.
    Переход $n - 1\Ra n$:
   \begin{align*}
       P(\xi_0 = a_0, \ldots, \xi_n = a_n) &= P(\xi_0 = a_0 , \ldots, \xi_{n - 1} = a_{n - 1})\cdot P(\xi_n = a_n | \xi_0 = a_0, \ldots, \xi_{n - 1} =
        a_{n - 1}) \\&= P(\xi_0 = a_0 , \ldots, \xi_{n - 1} = a_{n - 1})\cdot P(\xi_n = a_n | \xi_{n - 1} = a_{n - 1})  \\&=
        \pi_0(a_0)\cdot p_{a_0, a_1}\cdot\ldots\cdot p_{a_{n - 1}, a_n}.
   \end{align*}
\end{proof}

\begin{theorem}
    Если заданы $\pi_0\colon Y \Ra [0, 1]$ и $p\colon Y\times Y \Ra [0,1]$, такие что $\sum_{y\in Y}\pi_0(y) = 1$ и
    $\sum_{y\in Y} p_{xy} = 1$, то существует такое вероятностное пространство $(\Omega, \mathcal{F}, P)$ и цепь Маркова
    с начальным распределением $\pi_0$ и вероятностные переходы $p$.
\end{theorem}
\newpage
