\section{Усиленный закон больших чисел. Следствие. Метод Монте–Карло}

\begin{theorem}[усиленный закон больших чисел]
    Пусть $\xi_1, \ldots$ --- независимые случайные величины, $\EE \abs{\xi_k - \EE \xi_k}^4\le C$.
     Тогда $\frac{S_n}{n} - \EE \frac{S_n}{n} \rightarrow0$ почти наверное.
 \end{theorem}

 \begin{proof}
    Пусть $a_n =\EE \xi_n$, $\wt{\xi_n} = \xi_n - a_n$; очевидно, $\EE \wt{\xi_n} = 0$, и тогда нам надо доказать, что при
     $\EE \xi_n^4\le c $ верно $\frac{\wt{\xi_1} + \ldots + \wt{\xi_n}}{n}\rightarrow 0$ почти наверное.

     Далее считаем, что $a_n = 0$, $A_n = \left\{\abs*{\frac{S_n}{n}}>\e\right\}$.

     Если в какой то точке нет стремления к нулю, то это означает, что она лежит в бесконечном числе $A_n$.
     Мы раньше обсуждали, как можно описать все такие множества, они описываются как
     $$A = \overset{\infty}{\underset{n = 1}{\cap}}\overset{\infty}{\underset{k = n}{\cup}} A_k.$$ Таким образом, надо доказать, что $P(A) = 0$.
     По лемме \ref{lem:borkan}(Борелля-Кантелли) достаточно доказать, что $\sum_{n=1}^{\infty} P(A_n)< +\infty$.
     \begin{equation*}\label{eq:yzbc}\tag{$*$}
         P(A_n) = P\left(\abs*{\frac{S_n}{n}}^4 > \e^4\right) \le \frac{\EE (\frac{S_n}{n})^4}{\e^4} = \frac{\EE S_n^4}{n^4\e^4}.
     \end{equation*}
     Докажем, что $\EE s_n^4\le cn^4$.
     $$(\xi_1 + \ldots + \xi_n)^4 = \sum \xi_k^4 + C_1\underset{i\neq j}{\sum}\xi_i^2\xi_j^2 + C_2\sum\xi_i^2\xi_j\xi_k + C_3\sum\xi_i^3\xi_k + C_4\sum\xi_i\xi_j\xi_k\xi_l.$$
    Величины независимы, а $\EE \xi_k = 0$, так что
    \begin{align*}
        \EE  S_n^4 &= \sum \EE \xi_k^4 + C_1\underset{i\neq j}{\sum}\EE (\xi_i^2\xi_j^2)\\
         &\le nC + C_1\underset{i\neq j}{\sum}\EE (\xi_i^2)\EE (\xi_j^2) \le [\text{по неравенству Ляпунова}] \\
         &\le nC + C_1\sum\sqrt{\EE \xi_i^4}\sqrt{\EE \xi_j^4} \le nC + C_1n^2C \le cn^2.
    \end{align*}
    Применяя к (\ref{eq:yzbc}), получаем, что $P(A_n) \le \frac{c}{n^2\e^4}$, значит ряд сходится.
 \end{proof}

 \begin{corollary}[усиленный закон больших чисел схемы Бернулли] Пусть $p$~--- вероятность успеха в бернуллиевских величинах $\xi_i$.
     Тогда $\frac{S_n}{n}\rightarrow p$ почти наверное.
 \end{corollary}

 \begin{proof}
     $\EE (\xi_1 - p)^4<+\infty$, так как $(\xi_1 - p)^4$ принимает значение $p^4$ с вероятностью $1-p$ и $(1- p)^4$ с вероятностью $p$.
 \end{proof}


 \begin{example}[метод Монте-Карло]
     Есть фигура на плоскости. Хотим оценить её площадь. Для этого возьмём прямогульник, полностью покрывающий эту фигуру, и будем брать случайные точки внутри него. Пусть случайная величина $\xi_i = 1$, если $i$-ая случайная точка лежит внутри фигуры, и $\xi_i = 0$, если не лежит. Вероятность того, что точка попадёт? равна
     $p = \frac{\text{площадь фигуры}}{\text{площадь прямоугольника}}$.
    По следствию для схемы Бернулли, $\frac{S_n}{n}\rightarrow p$ почти наверное, то есть, посчитав большое количество точек, можно оценить площадь фигуры.
 \end{example}

 \begin{theorem}[усиленный закон больших чисел в форме Колмогорова] 
    Пусть $\xi_1, \ldots$ независимые одинаково распределенные случайные величины,
     $S_n = \xi_1 + \ldots + \xi_n$. Тогда $\frac{S_n}{n}\rightarrow a$ почти наверное равносильно тому, что $a=\EE \xi_1$.
 \end{theorem}
 