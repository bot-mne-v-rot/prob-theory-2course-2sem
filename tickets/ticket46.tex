\section{Распределение положений на $n$-м шаге. Стационарное распределение. Пример}

\begin{theorem}
    Пусть $\pi_n = P_{\xi_n}$, то есть распределение после $n$ шагов. Его можно представить как вектор
    длины $\abs{Y}$. $P$ --- матрица переходов, то есть матрица $\abs{Y} \times \abs{Y}$, элемент с координатами $(a, b)$ которой равен $p_{ab}$. Тогда
    $\pi_n = \pi_0 P^n$.
\end{theorem}

\begin{proof}
    База $n=0$ очевидна, матрица в нулевой степени --- единичная, получаем $\pi_0 = \pi_0$.
    Переход $n - 1\Ra n $: надо проверить, что $\pi_n = \pi_{n - 1}P$.
   Рассмотри элемент этого вектора, соответствующий значению $a \in Y$:
    $$P(\xi_n = a) = \underset{y\in Y}{\sum} P(\xi_{n - 1} = y)P(\xi_n = a | \xi_{n - 1} = y) = \underset{y\in Y}{\sum } \pi_{n - 1}(y) p_{ya}.$$
    Последнее --- это произведение $\pi_{n - 1}$ и столбца $P$, соответствующего $a$.
\end{proof}

\begin{definition}
   Распределение $\pi$ называется \textit{стационарным}, если $\pi P = \pi$.
\end{definition}

Из предыдущей теоремы становится ясно, что стационарное распределение --- то, которое не меняется со временем.

\begin{notation}
    Вероятность перехода из $a$ в $b$ за $n$ шагов 
   $$p_{ab}(n) = P(\xi_n = b | \xi_0 = a).$$
\end{notation}


\begin{example} Рассмотрим случайное блуждание на $\Z$.
    Выбираем сторону с вероятностью $\frac{1}{2}$. Пусть $\pi(y)$ --- стационарное распределение, тогда $$\frac{1}{2}\pi(y - 1) + \frac{1}{2}\pi(y+1) =
        \pi(y) \a \pi(y + 1) - \pi(y) = \pi(y) - \pi(y - 1)\Ra \pi(y + 1) - \pi (y) = \const. $$
    Эта константа не может не равняться нулю, так как иначе через некоторое количество шагов вероятность станет больше 1 или меньше 0, а такого быть не может.
    Значит, $\pi(y) = \pi(y + 1)$, то есть вероятность оказаться в точке для каждой точки одинакова, но такого тоже не бывает, поскольку мы знаем,
    что $\sum_{y\in Y} \pi(y) = 1$ и поэтому $\pi(y)$ не может не равняться нулю.
    Итого поняли, что случайное блуждание не имеет стационарного распределения.
\end{example}\newpage
