\section{Лемма Бореля–Кантелли. Закон нуля и единицы. Пример}

\begin{lemma}[Бореля-Кантелли]\label{lem:borkan}
    $A_1, A_2, \ldots$ --- последовательность случайных событий. Событие $B$ --- наступило бесконечное число из событий $A$.
     \begin{enumerate}
         \item  Если $\sum_{n = 1}^\infty P(A_n) < +\infty$, то $P(B) = 0$.
         \item Если $A_1, A_2, \ldots$ независимы в совокупности и $\sum_{n = 1}^\infty P(A_n) = +\infty$, то
    $P(B) = 1$.
     \end{enumerate}

\end{lemma}

\begin{proof}
     Поймём, что $B = \bigcap\limits_{n = 1}^\infty \bigcup_{k = n}^\infty A_k$. 
    $\omega \in B \Ra \omega \in A_{m_k} $ для некоторой последовательности $m_1,m_2, \ldots$ $\Rightarrow \omega$ лежит в каждом из объединений
    $\bigcup_{k = n}^\infty A_k$, то есть и в их пересечении.
    В обратную сторону:  $\omega \in \bigcap\bigcup \Ra \omega\in\bigcup_{k = n}^\infty A_k $ $\forall n$ $\Ra \omega\in A_{m_k}$ для
    некоторой последовательности $m_1, m_2, \ldots$, а это по определению $B$ означает, что $\omega\in B$.
   \begin{enumerate}
       \item По только что доказанному верно
       \begin{align*}
           P(B) &= P\left(\bigcap_{n = 1}^\infty \bigcup_{k = n}^\infty A_k\right) = [\text{из свойств меры}]
           \\ &= \underset{n\to \infty}{\lim}
        P\left(\bigcup_{k = n}^\infty A_k\right) \le \underset{n\to\infty}{\lim} \sum_{k = n}^\infty P(A_k) = 0.
       \end{align*}
       Последнее верно, так как это предел хвоста сходящегося ряда.
       \item По лемме \ref{ex:nez} $\ov{A_1}, \ov{A_2}, \ldots$ независимы в совокупности.
       \begin{gather*}
           P\left(\bigcap_{k = n}^\infty \ov{A_k}\right) \leftarrow P\left(\bigcap_{k = n}^m \ov{A_k}\right) = \prod_{k = n}^m P(\ov{A_k}) \rightarrow  \prod_{k = n}^\infty
        P(\ov{A_k}) =  \prod_{k = n}^\infty (1 - P(A_k)).\\
         \ln P\left(\bigcap_{k = n}^\infty \ov{A_k}\right) = \sum_{k = n}^\infty \ln (1 - P(A_k)) \le - \sum_{k = n}^\infty P(A_k)  = -\infty \Ra
        P\left(\bigcap_{k = n}^\infty \ov{A_k}\right) = 0.\\
        \ov{B} = \bigcup_{n = 1}^\infty\left(\ov{\bigcup_{k = n}^\infty A_k}\right) = \bigcup_{n = 1}^\infty\bigcap_{k = n}^\infty \ov{A_k}
        \Ra P(\ov{B}) \le \sum_{n = 0}^\infty P\left(\bigcap_{k = n}^\infty \ov{A_k}\right) = 0.
       \end{gather*}
        Значит, $P(B) = 1$. \qedhere
   \end{enumerate}
\end{proof}

\begin{corollary}[Закон 0 и 1 Колмогорова] Пусть $A_1, A_2, \ldots$ независимы в совокупности. Тогда вероятность того, что наступило бесконечное количество событий из $A$ равно $0$ или $1$.
\end{corollary}

\begin{example}
    Рассмотрим бесконечную орлянку: ОРРРРОРРООРР\ldots

    Посчитаем вероятность того, что <<OPPO>> встретилось бесконечное число раз. Событие $A_k$ --- эта последовательность встретилась, причём начиная с позиции $k$. Тогда $A_1, A_5, A_9, \ldots$ независимы в совокупности.
    $P(A_k) = p^2q^2$, $\sum_{n = 1}^\infty P(A_{4n+1}) = +\infty$. Тогда по лемме <<ОРРО>> случилось бесконечное число раз с вероятностью 1.
\end{example}\newpage
