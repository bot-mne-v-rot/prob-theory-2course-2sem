\section{Независимые события. Мотивировка и определение. Примеры. Попарная независимость и независимость в совокупности. Примеры}

\begin{definition}
    Случайные события $A$ и $B$ \textit{независимы}, если $P(A \cap B) = P(A)\cdot P(B)$.
\end{definition}
Равносильное определение:
$$P(A \cap B) = P(A) P(B) \iff P(B|A) = \frac{P(A\cap B)}{P(A)} = P(B).$$
\begin{definition}
    $A_1, \ldots, A_m$ \textit{независимы в совокупности}, если
    $$\forall{i_1, \ldots, i_k}\ P(A_{i_1}\cap\ldots\cap A_{i_k})=P(A_{i_1})\cdot\ldots\cdot P(A_{i_k}).$$
\end{definition}

\begin{lemma}\label{ex:nez}
    $A_1, \ldots , A_m$ независимы в совокупности $\Ra$
    $P(B_1\cap\ldots \cap B_m)=P(B_1)\cdot\ldots \cdot P(B_m)$, где $B_i = A_i$ или $\overline{A_i}$
\end{lemma}



Отметим, что независимость в совокупности неравносильна попарной независимости.
Приведём пример. Пространство~--- множество пар чисел при кидании двух кубиков.
Обозначим события $A$~--- чётное на первом, $B$~--- чётная на втором, $C$~--- чётная сумма.
\begin{gather*}
P(A) = P(B) = P(C) = \frac{1}{2}.\\
P(A\cap C) = P(B \cap C) = P(A \cap B) = \frac{1}{4}.\\
\end{gather*}
Значит, эти события попарно независимы. 
\begin{gather*}
P(A \cap B \cap C) = P(A \cap B) = \frac{1}{4}.\\
P(A)P(B)P(C) = \frac{1}{8},
\end{gather*}
то есть они не независимы в совокупности. 

\begin{definition}
    $B$ \textit{не зависит от совокупности событий} $A_1, \ldots, A_m$, если
    $$\forall{i_1, \ldots, i_k}\ P(B|A_{i_1},\ldots, A_{i_k})=P(B) \iff P(B \cap A_{i_1} \cap \ldots \cap A_{i_k}) = P(B)\cdot P(A_{i_1} \cap \ldots \cap A_{i_k}).$$
\end{definition}

\begin{theorem}[Эрдёша-Мозера]
   В турнире участвует $n$ волейбольных команд. Играют каждая с каждой, без ничей. Пусть $k$~--- наибольшее число, для которого всегда найдутся такие команды $a_1,\ldots, a_k$, что $a_i$ выиграла у $a_j$, если $i < j$. Тогда $k \leq 1 + [2\log_2{n}]$.

\end{theorem}
\begin{proof}
    Турнир~--- полный орграф (стрелочки от победителей к проигравшим). Подходящая цепочка~--- полный ациклический подграф. Пусть событие
    $A(a_1, \ldots, a_k)$~--- подошёл набор $a_1, \ldots a_k$. Тогда
    $P(A) = 2^{- {\binom{k}{2}}}$.
    Способов выбрать набор (выбрать $k$ команд и порядок на них) ~--- ${\binom{n}{k}} \cdot k!$.
Вероятность того, что какой-то набор подойдёт не превосходит
    $$2^{-{\binom{k}{2}}}\cdot \binom{n}{k}\cdot k!.$$
    Докажем, что если $k > 1 + [2\log_2{n}]$, то это значение меньше единицы, то есть существует граф, на котором нет такого набора. 
    \begin{gather*}
        k > 1 + [2\log_2{n}] \Ra \log_2 n < \frac{k - 1}{2} \Ra n < 2^{\frac{k-1}{2}}.\\
    2^{-{\binom{k}{2}}}\cdot {\binom{n}{k}}\cdot k! = 2^{-\frac{k \cdot (k-1)}{2}}\cdot \frac{n!}{(n-k)!}< 2^{-\frac{k \cdot (k-1)}{2}}\cdot n^k < 1.
    \end{gather*}
\end{proof}
