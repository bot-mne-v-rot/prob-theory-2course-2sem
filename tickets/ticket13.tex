\section{Примеры вероятностных распределений}

\begin{examples}[различных распределений]
    \enewline
        \begin{enumerate}
            \item Распределение Бернулли ($\xi\sim \text{Bern}(p)$).
                   \begin{gather*}
                       0\le p\le 1, \ \xi \colon \Om \to \{0, 1\},  \\
                       P(\xi = 0) = 1 - p,\ P(\xi = 1) = p.
                   \end{gather*}
   
            \item Биномиальное распределение ($\xi\sim \text{Binom}(p)$).
            \begin{gather*}
                \xi \colon \Omega \rightarrow \{0,1, \ldots, n\}, \\
                P(\xi = k) = \binom{n}{k}p^k(1-p)^{n - k}.
            \end{gather*}
   
            \item Распределение Пуассона ($\xi \sim \text{Pois}(\lambda)$).
            \begin{gather*}
                \lambda > 0, \ \xi \colon \Omega\rightarrow\{0, 1, 2, \ldots\},\\
                P(\xi = n) = \frac{\lambda^n e^{-\lambda}}{n!}.
            \end{gather*}
       
   
            \item Геометрическое распределение ($\xi\sim \text{Geom}(p)$).
            \begin{gather*}
                0< p < 1, \ \xi \colon \Omega\rightarrow \{1, 2, 3, \ldots\}, \\
                P(\xi = n) = p(1-p)^{n - 1}.
            \end{gather*}
   
            \item Дискретное равномерное распределение.
            \begin{gather*}
                \xi \colon \Omega\rightarrow \{a+1, \ldots, b\},\\
                P(\xi = n) = \frac{1}{b - a}.
            \end{gather*}
       
            \item Непрерывное равномерное распределение ($\xi \sim \mathcal{U}[a, b]$).
            \begin{gather*}
                \xi \colon \Omega\rightarrow [a, b],\\
                p_\xi (t) = \frac{1}{b - a}\mathbbm{1}_{[a, b]}(t).
            \end{gather*}
   
            \item Нормальное распределение ($\xi\sim \mathcal{N}(a, \sigma^2)$).
            \begin{gather*}
                a\in\mathbb{R}, \ \sigma > 0.\\
                p_\xi (t) = \frac{1}{\sqrt{2\pi}\sigma} e^{-\frac{(t-a)^2}{2\sigma^2}}.
            \end{gather*}
   
                  Для $\xi\sim\mathcal{N}(0, 1)$ это называется стандартным нормальным распределением. Нетрудно заметить, что функция распределения для такой величины равна $\Phi(x)$.
   
            \item Экспоненциальное распределение ($\xi\sim \text{Exp}(\lambda)$).
            \begin{gather*}
                \lambda > 0, \ \xi \colon \Omega \rightarrow [0, +\infty), \\
                p_\xi (t) =\lambda e^{-\lambda t}\mathbbm{1}_{[0, +\infty)}(t).
            \end{gather*}
        \end{enumerate}
    \end{examples}
   
   