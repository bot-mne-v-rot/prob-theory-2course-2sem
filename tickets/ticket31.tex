\section{Характеристическая функция. Свойства. Характеристическая функция нормального распределения}

\begin{definition} Характеристическая функция вещественнозначной случайной величины $\xi$ --- это
    $$\phi_\xi(t) = \EE e^{i\xi t}.$$
\end{definition}

\begin{properties}[характеристической функции]
\enewline
    \begin{enumerate}
        \item $\phi_\xi(0) = 1$, $\abs{\phi_\xi(t)}\le 1$;

        \item $\phi_{a\xi + b}(t) = e^{itb}\phi_\xi (at)$;
             
        \item Если $\xi$ и $\eta$ независимы, то $\phi_{\xi + \eta}(t) = \phi_\xi(t)\phi_\eta(t)$

        \item Если $\xi_1, \ldots, \xi_n$ независимы, то $\phi_{\xi_1 + \ldots + \xi_n}(t) = \phi_{\xi_1}(t)\cdot\ldots\cdot \phi_{\xi_n}(t)$

        \item $\phi_\xi(-t) = \overline{\phi_\xi(t)}$

        \item $\phi_\xi(t)$~--- равномерно непрерывна на $\mathbb{R}$

    \end{enumerate}
\end{properties}

\begin{proof}
\enewline
\begin{enumerate}
    \item[2.] $\phi_{a\xi + b}(t) = \EE (e^{it(a\xi + b)}) = e^{itb}\EE e^{ai\xi t} = e^{itb}\phi_\xi(at)$
    \item[3.]  $\phi_{\xi + \eta}(t) = \EE (e^{it(\xi + \eta)}) = \EE (e^{it\xi}\cdot e^{it\eta}) = \EE e^{it\xi}\EE e^{it\eta} = \phi_\xi(t)\phi_\eta(t)$
    \item[6.]  $\abs{\phi_\xi(t + h) -\phi_\xi(t)} = \abs{\EE (e^{i\xi(t+h))} - e^{it\xi})} = \abs{\EE (e^{it\xi}(e^{ih\xi} - 1))} \le \EE \abs{e^{ih\xi} - 1}.$
    Хотим доказать, что $\EE \abs{e^{ih\xi} - 1} \to 0$ при $h \to 0$. По определению матожидания
    $$\EE \abs{e^{ih\xi} - 1} = \int\limits_\mathbb{R}\abs{e^{ihx} - 1}\dd P_\xi(x).$$
    $\abs{e^{ihx} - 1} \to 0$. По теореме Лебега подынтегральное выражение всегда $\le 2$, то есть это суммируемая мажоранта и можно менять предел и интеграл местами. ,
    \qedhere
\end{enumerate}
\end{proof}


\begin{example} Пусть $\xi\sim \mathcal{N}(a, \sigma^2)$, найдём его производящую функция $\phi_\xi (t)$.

    Пусть $\eta \sim \mathcal{N}(0, 1)$; знаем, что $\xi = \sigma\eta + a$, $\phi_\xi(t) = e^{ita}\phi_\eta(\sigma t)$,
    поэтому достаточно найти характеристическую функциею для $\eta$.
    $$\phi_\eta(t) = \frac{1}{\sqrt{2\pi}}\int\limits_\mathbb{R}e^{itx}e^{-\frac{x^2}{2}}\dd x =
        \frac{e^{-\frac{t^2}{2}}}{\sqrt{2\pi}}\int\limits_\mathbb{R}e^{-\frac{(x - it)^2}{2}}\dd x.$$

    Знаем, что $\intt_\mathbb{R}f(x) \dd x = \sqrt{2\pi}$, где $f(z) = e^{-\frac{z^2}{2}}$, хотим посчитать 
    $I = \int\limits_\mathbb{R}f(x-it)\dd x.$
    Сделаем это с помощью вычетов. Для этого посчитаем интеграл по контуру $\Gamma_R$:

     $$0 = \int\limits_{\text{Г}_R}f(z)\dd z = \int\limits_{-R}^R + \int\limits_R^{R - it} + \int\limits_{R - it}^{-R - it} + \int\limits_{-R - it}^{-R}.$$
     Знаем, что
     $$\int\limits_{-R}^R \to \sqrt{2\pi}, \ \int\limits_{R - it}^{-R - it}\to -I.$$
     Оценим второй: 
     $$\abs*{\int\limits_{R}^{R - it}f(z)\dd z} \le \int\limits_0^{-t}\abs{e^{-\frac{(R - iy)^2}{2}}}\dd y =
        \int\limits_0^{-t}(e^{-\frac{R^2}{2} + \frac{y^2}{2}})\dd y \le t\cdot \max (e^{-\frac{R^2}{2} + \frac{y^2}{2}}) =
        t\cdot e^{\frac{t^2}{2}}\cdot e^{-\frac{R^2}{2}} \to 0.$$ 
   Аналогично с последним интегралом. Итого получаем, что
    $$I = \sqrt{2\pi}\Ra\phi_\eta(t) = \frac{e^{-\frac{t^2}{2}}}{\sqrt{2\pi}}\sqrt{2\pi} =
        e^{-\frac{t^2}{2}} \Ra\phi_\xi(t) = e^{ita}e^{-\frac{\sigma^2t^2}{2}}.$$
\end{example}\newpage
