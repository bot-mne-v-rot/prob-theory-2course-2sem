\section{Ветвящиеся процессы. Вероятность вырождения}

Модель следующая. В начальный момент времени есть одна одна частица. Далее на каждом шаге каждая частица с вероятностью $f_k$ делится на $k$ частиц, причём
 $\sum_{k=0} ^ {+\infty} f_k = 1$.
Обозначим за $\eta_i$ количество частиц на $i$-м шаге, а $\xi_j^(i)$ --- количество потомков $j$-ой частицы (на каждом шаге своя нумерация) на $i$-м шаге. Тогда
\begin{gather*}
    \eta_0 = 1, \\
    \eta_1 = \xi_1^{(1)}, \\
    \eta_2 = \xi_1^{(2)} + \ldots + \xi_{\eta_1}^{(2)}, \ldots,
\end{gather*}
 причем $P(\xi_j^{(n)} = k) = f_k$.

 Пусть $G_n(z)$~--- производящая функция для $\eta_n$,
 $G(z)$~--- производящая функция для $\xi_i^{(k)}$. Из написанного выше $G_1 = G$.
 $$G_n(z) = \EE z^{\eta_n} = \sum\limits_{k=0} ^ {+\infty}  P(\eta_{n - 1} = k)\EE z^{\xi_1^{(n)} + \ldots + \xi_k^{(n)}} = \sum\limits_{k=0} ^ {+\infty}
     P(\eta_{n - 1} = k)G^k(z) = G_{n - 1}(G(z)).$$
 Поэтому $G_n(z) = \underbrace{G(G(\ldots (G}_{n \text{ раз}}(z))))$.

 Таким образом, поняли как устроена производящая функция, теперь посчитаем матожидание числа частиц:
 $$\EE \eta_n = G_n^\prime(1) = G_{n - 1}^\prime(G(1))\cdot G^\prime(1) = G_{n - 1}^\prime(1)\cdot G^\prime(1) = [\text{ по индукции }] = (\EE \eta_1)^n.$$


 \begin{theorem}
     Вероятность вырождения процесса~--- наименьший неотрицательный корень уравнения $G(x) = x$.
 \end{theorem}

 Заметим, что
 \begin{gather*}
     G(x) = \sum\limits_{k=0} ^ {+\infty} f_kx^k; \\
     G^\prime(x) = \sum\limits_{k=1} ^ {+\infty} kf_kx^{k - 1}\ge 0 \fpri x\in [0, 1]; \\
     G^{\prime \prime}(x) = \sum\limits_{k=2} ^ {+\infty} k(k - 1)f_kx^{k - 2}\ge 0\fpri x\in [0, 1].
 \end{gather*}
     Таким образом, функция монотонна и выпукла.
 \begin{proof}
     Обозначим
     $A_n = \{\eta_n = 0\}$, очевидно $A_n \subset A_{n + 1}$. Также понятно, что
     $P(A_n) = G_n(0) \le 1$.
     Раз события вложены, то вероятности неубывают и эти вероятности само собой ограничены, а значит есть предел, обозначим его за
     $q = \lim P(A_n) = \lim G_n(0)$.

     C одной стороны, $$G_{n + 1}(0) \to q.$$
     С другой, как мы выяснили выше,
     $$G_{n + 1}(0)= G(G_n(0))\to G(q).$$ 
     Получается $q = G(q)$, то есть найденный предел --- корень
     уравнения $G(x) = x$.

     Осталось доказать, что $q$~--- наименьший корень. Пусть $y$~--- наименьший неотрицательный корень. Тогда докажем, что $G_n(0)\le y$ всегда, тогда в
     пределе получим, что $q \le y$, доказав то, что нужно.
По условию $0\le y$, значит
$$G(0)\le G(y) = y \Ra G_2(0) \le G(y) = y$$ и так до n, получаем $G_n(0) \le y$.
 \end{proof}