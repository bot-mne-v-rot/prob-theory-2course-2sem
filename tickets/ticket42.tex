\section{Математическое ожидание и производящая функция суммы случайного количества случайных величин}

\begin{example}
    Пусть $N, \xi_1, \xi_2, \ldots$ --- независимые случайные величины, $\xi_1, \xi_2, \ldots$ одинаково распределены, $\EE \xi_1 = a$; обозначим
    $S = \xi_1 + \ldots + \xi_N$. Найдем $\EE S$:
    \begin{align*}
        \EE S &= \EE (\EE (S | N)) = \EE (\sum \EE (S | N = n)\mathbbm{1}_{\{N = n\}}) = \sum \EE (S | N = n) P(N = n) \\&= \sum \EE (\xi_1 + \ldots + \xi_n) P(N = n) = \sum na P(N = n) =
        a\sum nP(N= n) = a\EE N.
    \end{align*}

\end{example}

\begin{exercise}
    Найдите $\DD S$, если известно $\DD\xi_1, \DD N, \EE \xi_1, \EE N$.
\end{exercise}

\begin{example}
   Пусть случайные величины $N, \xi_1, \xi_2, \ldots$ принимают натуральные значения.
    $G$ ~--- производящая функция $N$, $F$ ~--- производящая функция $\xi_1$. Найдем производящую функцию для $S$:
    \begin{align*}
        \EE z^S &= \EE (\EE (z^S | N)) = \sum \EE (z^S | N = n)P(N = n) = \sum \EE (z^{\xi_1 + \ldots + \xi_n}) P(N = n) \\&=
        \sum \EE z^{\xi_1}\cdot\ldots\cdot \EE z^{\xi_n} P(N = n) = \sum (F(z))^n P(N= n) = G(F(z)).
    \end{align*}
   
\end{example}

\begin{remark}[геометрическая интерпретация]
Рассмотрим $\xi$, такие что $\EE \xi^2 < +\infty$. Они образуют пространство $L^2(\Omega, \mathcal{F}, P)$; возьмём $\s$-алгебру $\mathcal{A} \subset\mathcal{F}$. Условные матожидания живут в пространстве $L^2(\Omega, \mathcal{A}, P)\subset L^2(\Omega, \mathcal{F}, P)$. Тогда 
условное матожидание $\EE (\xi |\mathcal{A} )$ --- проекция $\xi$ на $L^2(\Omega, \mathcal{A}, P)$.
\end{remark}


\begin{proof} Нужно доказать, что
   $$\xi- \EE (\xi | \mathcal{A} ) \perp L^2(\Omega, \mathcal{A}, P).$$ 
   Достаточно проверить на $\mathbbm{1}_A$ для $A\in\mathcal{A}$, то есть проверить, что
    $$\EE ((\xi - \EE (\xi | \mathcal{A}))\mathbbm{1}_A) = 0,$$ что равносильно тому, что
    $$\EE (\xi \mathbbm{1}_A) = \EE (\EE (\xi | \mathcal{A})\mathbbm{1}_A,$$
    а что в свою очередь является определением.
\end{proof}

\begin{theorem}
\enewline
    \begin{enumerate}
        \item Если $\xi, \eta$ независимы, то $\EE (\xi | \eta) = \EE \xi$
        \item Если $\eta$ измерима относительно $\mathcal{A}$, то $\EE (\xi\eta | \mathcal{A}) = \eta \EE (\xi | \mathcal{A})$
    \end{enumerate}
\end{theorem}


\begin{proof}
\enewline
    \begin{enumerate}
        \item Проверим, что $\EE \xi$ подходит, то есть то, что 
        $$\EE (\EE \xi\mathbbm{1}_A) = \EE (\xi \mathbbm{1}_A) \quad \forall A\in \sigma(\eta).$$
              $\sigma(\eta)$ натянута на $\{\eta \le c\}$, достаточно проверить для $A = \{\eta \le c\}$.
              Надо показать, что
              $$\EE (\EE \xi\mathbbm{1}_A) = \EE \xi \EE \mathbbm{1}_A = \EE (\xi\mathbbm{1}_A)$$ 
              Первое верно, так как $\mathbbm{1}_A$ --- константа, а второе 
              верно, если $\xi $ и $\mathbbm{1}_A$ независимы, то есть когда события $\{\xi \le a\}$ и $\{\mathbbm{1}_A \le b\}$ независимы (заметим, что это верно, поскольку $\{\mathbbm{1}_A \le c\} = \emptyset$ или $A = \{\eta \le c\}$).

        \item Докажем по стандартной схеме(проверяем для индикаторной функции, по линейности верно для простых, приближаем произвольную простыми и переход к пределу по теореме Леви).
        Проверяем для $\eta = \mathbbm{1}_B$ при $B \in \mathcal{A}$. Надо проверить, что
              $$\EE (\xi\mathbbm{1}_B | \mathcal{A}) = \mathbbm{1}_B \EE (\xi | \mathcal{A}),$$ то есть то, что 
              $\mathbbm{1}_B \EE (\xi | \mathcal{A})$ подходит, что равносильно тому, что для любого $A\in \mathcal{A}$ у нас есть равенство
              $$\EE (\mathbbm{1}_B \EE (\xi | \mathcal{A}) \mathbbm{1}_A) = \EE (\xi \mathbbm{1}_B\mathbbm{1}_A).$$
              Оно есть, так как 
              $$\EE (\mathbbm{1}_B \EE (\xi | \mathcal{A}) \mathbbm{1}_A)  = \EE (\EE (\xi | \mathcal{A} ) \mathbbm{1}_{A\cap B}) = \EE (\xi\mathbbm{1}_{A\cap B}) = \EE (\xi \mathbbm{1}_B\mathbbm{1}_A).$$ \qedhere
    \end{enumerate}
\end{proof}
\newpage
