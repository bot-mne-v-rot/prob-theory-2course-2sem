\section{Эргодическая теорема Маркова}

\begin{theorem}[Маркова]
    Пусть пространство состояний $Y$ конечно и $p_{ab} > 0$ для всех $a, b\in Y$, тогда существует единственное $\pi$ --- стационарное распределение, причём
    $\pi(b) = \lim_{n\to+\infty}p_{ab}(n)$. Более того,
    существуют $c > 0$ и $\lambda \in (0, 1)$ такие,
    что 
    $$|\pi(b) - p_{ab}(n)|\le c\lambda^n\quad \forall a, b\in Y.$$ 
\end{theorem}

Стоит отметить, что условие не зависит ни от начального распределения, ни начальной позиции. 

\begin{proof}
   Вспомним про теорему Банаха о сжатии: если есть $(X, \rho)$ -- полное метрическое пространство и $T:X \to X$ -- сжимающее отображение с коэффициентом $\lambda \in (0, 1)$ (т.е. это такое отображение, что $\rho(T(x), T(y)) \le \lambda \cdot \rho(x, y)$), тогда существует единственная неподвижная точка (т.е. такой $x'$, что $T(x')=x'$). Сведем нашу теорему к этой.
   
   В качестве полного метрического пространства возьмем $\R^d$, где $d$-- количество элементов в $Y$, а в качетве нормы: $\parallel x \parallel  := |x_1| + \dots + |x_d|$. Рассмотрим $X = \{ x \in \R^d : \parallel x \parallel = 1, x_i \ge 0 \}$ -- это соответствует всем распределениям. В качестве отображения логично взять умножение на матрицу перехода $T(x) := xP$. Проверим, что оно сжимающее.
   
   Пусть $z := y - x$, тогда нам надо проверить, что: \begin{gather*}
       \parallel T(y) - T(x) \parallel \le \lambda \parallel y - x \parallel \\
       \parallel T(z) \parallel \le \lambda \parallel x \parallel
   \end{gather*}
   Важно, что у $z$ сумма координат 0.
   Пусть $\delta := \min\limits_{a, b, \in Y} p_{ab} > 0$.
   Оценим $\parallel T(z) \parallel$: \begin{align*}
        \parallel T(z) \parallel &= \sum_{k=1}^d |(T(z))_k| \\
        &= \sum_{k=1}^d |\sum_{j=1}^d z_jp_{jk}| (\text{вспомним что } z_1 + \dots + z_d = 0) \\
        &= \sum_{k=1}^d |\sum_{j=1}^d z_j(p_{jk} - \delta)|  \\
        &\le \sum_{k=1}^d \sum_{j=1}^d 
        |z_j|(p_{jk} - \delta) \\
        &= \sum_{j=1}^d |z_j| \sum_{k=1}^d 
        (p_{jk} - \delta) (\text{вспомним что } p_{1k} + \dots + p_{kk} = 1) \\
        &= (1-d\delta)\sum_{j=1}^d |z_j| = (1-d\delta)\parallel z \parallel
   \end{align*}
   Осталось сказать, что $(1-d\delta)$ и есть $\lambda$ из теоремы Банаха о сжатии. Скорость сходимости тоже следует из теоремы Банаха.
\end{proof}

\begin{corollary}
    Пусть $Y$ --- конечное множество, и для некоторого $n$ выполняется
    $$p_{ab}(n) > 0 \quad \forall a, b\in Y.$$
    Тогда существует единственное стационарное распределение $\pi$, такое что
    $$\underset{m\to+\infty}{\lim} p_{ab}(m) = \pi(b)\quad \forall a, b\in Y.$$
\end{corollary}
