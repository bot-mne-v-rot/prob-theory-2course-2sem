\section{Случайные блуждания на Z (в соседние точки и произвольное симметричное)}

\begin{theorem} \label{th:Zvoz}
    Случайное блуждание на $\mathbb{Z}$ возвратно тогда и только тогда, когда $p = \frac{1}{2}$ (то есть оно симметрично).
\end{theorem}

\begin{proof}
Воспользуемся критерием возвратности.
    $$\sum\limits_{n=1} ^ {+\infty} p_{00}(n) = \sum\limits_{n=1} ^ {+\infty} p_{00}(2n) = \sum\limits_{n=1} ^ {+\infty} \binom{2n}{n}p^n (1 - p)^n.$$
   Знаем формулу(например, как следствие формулы Стирлинга)
    $$\binom{2n}{n}p^n(1-p)^n \sim \frac{(4p(1-p))^n}{\sqrt{\pi n}}.$$
   Тогда
    если $ p \neq\frac{1}{2}$ $4p(1-p) < 1$, то ряд сходящийся, и по критерию блуждание невозвратное.
Если $ p =\frac{1}{2}$ $\binom{2n}{n}p^n(1-p)^n \sim \frac{1}{\sqrt{\pi n}}$, то ряд расходится и блуждание возвратное.
\end{proof}

Теперь опишем произвольное симметричное случайное блуждание (то есть можем перейти не только на соседние клетки) на $\mathbb{Z}$ следующим образом.
$\xi_1, \xi_2, \ldots$ --- независимые одинаково распределенные симметричные целочисленные случайные величины. Обозначим
$S_n = \xi_1 + \ldots + \xi_n$.

\begin{theorem}
    Если $\xi_k$ симметричные и имеют матожидание, то случайное блуждание возвратно.
\end{theorem}



Теперь рассмотрим случайные блуждания в $\mathbb{Z}^d$, вероятность перехода  в каждую сторону равна $\frac{1}{2d}$.
