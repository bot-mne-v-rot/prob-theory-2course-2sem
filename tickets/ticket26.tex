\section{Закон больших чисел. Следствия}

\begin{theorem}[закон больших чисел] Пусть величины
    $\xi_1, \ldots$ попарно некоррелированы, для всех $i$ верно
    $\DD\xi_n < M$, $S_n = \xi_1 + \ldots + \xi_n$. Тогда 
    $$P\left(\abs*{\frac{S_n}{n} - \EE \frac{S_n}{n}}\ge t\right)\rightarrow 0\fpri t > 0.$$
\end{theorem}

\begin{proof}
   Применим неравенство Чебышёва:
   \begin{align*}
       P\left(\abs*{\frac{S_n}{n} - \EE \frac{S_n}{n}}\ge t\right) &\le  \frac{\DD(\frac{S_n}{n})}{t^2} = \frac{\DD(S_n)}{n^2t^2} = [\cov(\xi_i, \xi_j)=0] \\&=  \frac{\sum\DD\xi_n}{n^2t^2} \le \frac{nM}{n^2t^2} = \frac{M}{nt^2}\rightarrow 0.
   \end{align*}
    
\end{proof}

\begin{corollary}[закон больших чисел в форме Чебышёва] 
Пусть $\xi_1, \ldots$ --- независимые одинаково распределенные случайные величины с конечной дисперсией и $a = \EE \xi_1$, тогда 
$$P\left(\abs*{\tfrac{S_n}{n} - a}\ge t\right) \rightarrow 0 \fpri t > 0,$$ 
то есть $\frac{S_n}{n}$ сходится к $a$ по вероятности.


\end{corollary}
\begin{proof}
                  Величины независимы, поэтому некоррелированы, дисперсии ограничены и, поскольку матожидание суммы равно сумме матожиданий (которые равны $a$, поскольку они одинаково распределены), $\EE \frac{S_n}{n} = a$; применим теорему.
              \end{proof} 
\begin{corollary}[закон больших чисел для схем Бернулли]
              Пусть $\xi_1, \ldots$ независимые бернуллиевские случайные величины с вероятностью $p$. Тогда 
              $$P\left(\abs*{\frac{S_n}{n} - p}\ge t\right)\rightarrow 0 \fpri t > 0.$$
             
  
\end{corollary}
              \begin{proof} Для бернуллиевских величин верно
              \begin{gather*}
                  \EE \xi_1 = P(\xi_1 = 1) = p;\\
                  \DD\xi = \EE \xi_1^2 - (\EE \xi_1)^2 = p - p^2.
              \end{gather*}
        Показали ограниченность дисперсий, и можно применить теорему.
    \end{proof}
\newpage
