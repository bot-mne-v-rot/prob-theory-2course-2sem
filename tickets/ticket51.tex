\section{Задача о разорении}

\begin{example}[задача о разорении]

    Два игрока, у которых $A$ и $B$ монет соответственно играют в орлянку, с вероятностью $q=1-p$ первый платит второму и с вероятностью $p$
      --- второй первому.
      Найдем вероятность разорения, обозначим за $\beta_k(x)$ вероятность на $k$-ом шаге
      оказаться в $B$, если на нулевом шаге мы находимся в точке $x$, $-A < x < B$.
      \begin{gather*}
          \beta_k(x) = p\beta_{k - 1}(x + 1) + q\beta_{k -1}(x-1);\\
          \beta_{k - 1}(x) \le \beta_k(x) \le 1.
      \end{gather*}
     Тогда существует 
     $\lim_{k\rightarrow\infty}\beta_k(x)$ обозначим его за $\beta(x) \le 1.$
      Получили
      \begin{gather*}
          \beta(x) = p\beta(x - 1) + q\beta(x +1);\\
          \beta(-A) = 0, \ \beta(B) = 1.
      \end{gather*}
 
      Нужно решить это соотношение.
      Если $p\neq q$, то $pt^2 - t + q = 0$, $t_1 = 1, t_2 = \frac{q}{p}$ ---  его корни.
 
      Тогда $at_1^n + bt_2^n$ ~--- решение соотношения, потому что 
      $$pt_1^n = pt_1^2\cdot t_1^{n - 2} =
          (t_1 - q)t_1^{n - 2} = t_1^{n - 1} - qt_1^{n - 2}.$$
     Таким образом,
      $\beta(x) = a + b(\frac{q}{p})^x$, осталось подобрать $a, b$ так, чтобы совпали значения в $-A, B$. Итого $$\beta(x) = \frac{(\frac{q}{p})^x - (\frac{q}{p})^{-A}}{(\frac{q}{p})^B - (\frac{q}{p})^{-A}}.$$
     
  \end{example}