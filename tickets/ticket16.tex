\section{Свертки мер. Свертки мер, имеющих плотность}

В этом паграфе $\mu$ и $\nu$ --- конечные меры на борелевских подмножествах $\mathbb{R}$.
    
     \begin{definition} \textit{Сверткой мер} $\mu$ и $\nu$ называется
         $$\mu * \nu(A) = \int\limits_\mathbb{R} \mu(A - x) \dd\nu(x).$$
     \end{definition}

     \begin{properties}[свёртки мер]
    \enewline
         \begin{enumerate}

             \item $\mu * \nu(A) = \intt_{\mathbb{R}^2}\mathbbm{1}_{A}(x+y) \dd\mu(x)\dd\nu(y)$;

             \item $\mu*\nu = \nu*\mu$;

             \item $(\mu_1*\mu_2)*\mu_3 (A)  = \intt_{\mathbb{R}^3}\mathbbm{1}(x_1+x_2+x_3)\dd\mu_1(x_1)\dd\mu_2(x_2)\dd\mu_3(x_3) = \mu_1*(\mu_2*\mu_3) (A)$;

             \item $\mu_1*\mu_2*\ldots*\mu_n (A)  = \intt_{\mathbb{R}^n}\mathbbm{1}(x_1+x_2+\ldots +x_n)\dd\mu_1(x_1)\dd\mu_2(x_2)\ldots \dd\mu_n(x_n)$;

             \item $(c\mu)*\nu = c\cdot\mu*\nu$;

             \item $(\mu_1+\mu_2)*\nu = \mu_1 * \nu + \mu_2 * \nu$;

             \item Пусть $\delta_a$ --- мера, такая что $\delta_a(\{a\}) = 1$ и $\delta_a(\{\mathbb{R}\setminus \{a\}\}) = 0$. Тогда
                   $\mu*\delta_0 = \mu$ (то есть $\delta_0$ --- единица с точки зрения свертки).

                   
         \end{enumerate}
     \end{properties}
     
      \begin{proof}
\enewline
         \begin{enumerate}
             \item По определению свёртки
             \begin{align*}
                 \mu * \nu(A) &= \int\limits_\mathbb{R} \mu(A - x) \dd\nu(x) \\&= \int\limits_{\mathbb{R}}\int\limits_\mathbb{R} \mathbbm{1}_{A - y}(x) \dd\mu(x)\dd\nu(y) \\&= \int\limits_{\mathbb{R}^2}\mathbbm{1}_A (x + y) \dd\mu(x)\dd\nu(y).
             \end{align*}
             \item[] Пункты 2-6 очевидно следуют из определения и пункта 1.
             \item[7.] По определению свёртки
             $$\mu*\delta_0 (A) = \int\limits_\mathbb{R} \mu(A - x)d\delta_0(x) =  \mu (A - 0) = \mu A.$$ \qedhere
         \end{enumerate}
                       
\end{proof}

     \begin{theorem}
         Пусть $p_\mu$ и $p_\nu$ --- плотности мер $\mu$ и $\nu$ относительно меры Лебега $\lambda$.
         Тогда $\mu*\nu$ имеет плотность
         $$p(t) = \int\limits_\mathbb{R} p_\mu (t - x) p_\nu (x) \dd x.$$
         Это называется \textit{свёрткой функцией}.
     \end{theorem}

     \begin{proof}
        Надо проверить, что 
        $$\mu*\nu(A) = \int\limits_A p(t)\dd t.$$ 
        По определению $p(t)$
        \begin{align*}
            \int\limits_A p(t)\dd t &= \int\limits_A\int\limits_\mathbb{R} p_\mu(t - x)p_\nu(x)\dd x\dd t \\&= \int\limits_{\mathbb{R}^2}\mathbbm{1}_A(t)p_\mu(t-x)p_\nu(x) \dd x\dd t \\&= \int\limits_{\mathbb{R}^2}\mathbbm{1}_A(x+y)p_\mu(y)p_\nu(x)\dd x\dd y \\&= \int\limits_{\mathbb{R}^2} \mathbbm{1}_A(x+y) \dd\mu(y)\dd\nu(x) = \mu*\nu(A).
        \end{align*}
       
     \end{proof}
