\section{Схема Бернулли. Полиномиальная схема. Теорема Эрдёша–Мозера}

\begin{definition} \textit{Схема Бернулли}. Элементарные события в пространстве имеют вид $\om = (x_1, \ldots, x_n)$, где $x_i$ принимает значение $0$ или $1$; определим вероятность как
    $$P(\omega) = p^{\sum x_i}q^{n - \sum x_i},\ \text{ где } \ p \in [0, 1], \ q = 1-p.$$ 
    \end{definition}
   
   Смысл такой --- рассмотрим модель, в которой мы делаем $n$ подбрасываний монетки, при которых орёл выпадает с вероятностью $p$, а решка с вероятность $q=1-p$. Тогда элементарные события --- это все возможные исходы; нетрудно проверить, что вероятность исхода совпадает с вероятностью из определения. 
   
   Подсчитаем вероятность выпадения $k$ орлов. Надо сложить вероятность по всем возможным $\om$, в которых ровно $k$ орлов(они имеют одинаковые вероятности).
   $$P(k \text{ орлов}) = \binom{n}{k}p^kq^{n - k}.$$
   
    \begin{definition}\textit{Полиномиальная схема}.
   Элементарные события в пространстве имеют вид $\om = (x_1, \ldots, x_n)$, где $x_i$ принимают целочисленные значения от 1 до $m$; определим вероятность как
   $$P(\omega) = p_1^{\#\{i\mid x_i = 1\}}\cdot p_2^{\#\{i\mid x_i = 2\}}\cdot\ldots\cdot p_m^{\#\{i\mid x_i = m\}},\ \text{где} \ p_i \ge 0, \ \sum p_i = 1.$$
    \end{definition}
   Рассмотрим модель, аналогичную предыдущей, но $x_i$ принимает значение $k$ с вероятностью $p_k$.
   \begin{gather*}
       P(k_1 \text{ раз } 1, \ldots, k_m \text{ раз } m) = p_1^{k_1}\cdot\ldots\cdot p_m^{k_m}\cdot \binom{n}{k_1,\ldots, k_m}, \ \text{где} \\ \binom{n}{k_1,\ldots, k_m} = \frac{n!}{k_1!\ldots k_m!} \ \text{--- мультиномиальный коэффициент.}
   \end{gather*}
    