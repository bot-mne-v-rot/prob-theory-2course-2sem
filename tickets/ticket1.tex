\section{Конечное вероятностное пространство. Свойства вероятности. Классическое определение вероятности}

\begin{definition}
    $\Om = \{\om_1, \om_2, \ldots, \om_n\}$~--- \textit{множество (пространство) элементарных событий}, если
    \begin{enumerate}
        \item $\om_i$ -- равновозможны;
        \item $\om_i$ и $\om_j$ не реализуемы одновременно (несовместны);
        \item Какая-то $\om_i$ случается;
    \end{enumerate}
\end{definition}

\begin{examples}
   \enewline
    \begin{enumerate}
        \item Монетка~--- орёл или решка($1/0$);
        \item Игральный кубик;
        \item Колода карт;
    \end{enumerate}
\end{examples}
Здесь и далее за $\#A$ обозначается мощность множества $A$.
\begin{definition}
   \textit{Случайное событие} --- это некоторое $A \subset \Om$.
   \textit{Вероятность случайного события} --- это $P(A)= \frac{\#A}{\#\Om}$.
\end{definition}

Для примеров пространств элементарных событий приведём примеры случайных событий: 
\begin{examples}
\enewline
    \begin{enumerate}
        \item Орёл/решка;
        \item Чётное число очков/число очков больше трёх;
        \item Пики/красные старше валета;
    \end{enumerate}
\end{examples}

\begin{properties}[вероятности]
    \begin{enumerate}
        \item $P(\emptyset) = 0, \; P(\Om) = 1$;
        \item Если $A \cap B = \emptyset$ (говорят, что они \textit{несовместны}), то $P(A \cup B) = P(A) + P(B)$;
        \item $P(A \cup B) = P(A) + P(B) - P(A \cap B)$;
        \item $P(A \cup B) \leq P(A) + P(B)$;
        \item \textit{Формула включений-исключений}:
        \begin{multline*}
             P(A_1\cup A_2 \cup \ldots \cup A_n) = \sum_i{P(A_i)} - \sum_{i<j}P(A_i\cap A_j) \\+ \sum_{i<j<k}P(A_i\cap A_j \cap A_k) - \ldots +(-1)^{n-1}P(A_1\cap\ldots\cap A_n).
        \end{multline*}
    
        \item $P(\ov{A}) = 1 - P(A) = P(\Om) - P(A)$, где $\ov{A} = \Om \setminus A$;
    \end{enumerate}
\end{properties}

\begin{proof}
\begin{enumerate}
    \item[5.]Индукция. База --- пункт 3. Переход $n \to n+1$.
   Обозначим $B = A_1 \cup \ldots \cup A_n$. Тогда по 3-му пункту
    $$P(B\cup A_{n+1}) = P(B) +  P(A_{n+1}) - P(B\cap A_{n+1}).$$
   Заметим, что 
   $$B\cap A_{n+1} = \bigcup_{i=1}^n(A_i\cap A_{n+1}).$$
   Тогда по индукционному переходу
   $$P(B\cap A_{n+1}) = \sum_{i\leq n}P(A_i\cap A_{n+1}) - \sum_{i<j\leq n}P(A_i\cap A_j \cap A_{n+1}) + \ldots.$$
   \item[6.] Следствие из первого и третьего. \qedhere
\end{enumerate}
\end{proof}\newpage
