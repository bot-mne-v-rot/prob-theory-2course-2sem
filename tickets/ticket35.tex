\section{Теорема о сходимости по распределению (все, кроме 7 $\Rightarrow$ 1)}

\begin{theorem}
    Пусть $\xi_1, \xi_2, \ldots$ последовательность случайных величин, $F_1, F_2, \ldots$~--- их функции распределения, $\phi_1, \phi_2, \ldots$~--- их характеристические функции. Тогда следующие условия равносильны:

     \begin{enumerate}
         \item $\xi_n \rightarrow \xi$ по распределению;

         \item  Для любого открытого $ U \subset\mathbb{R}$ верно $$\varliminf_{n\rightarrow+\infty} P(\xi_n\in U ) \ge P(\xi\in U ).$$

         \item Для любого замкнутого $A\subset\mathbb{R}$ верно
         $$\varlimsup_{n\rightarrow+\infty} P(\xi_n\in A) \le P(\xi\in A).$$

         \item Для любого борелевского регулярного множества $B$ (то есть $P(\xi \in \Cl B \setminus \Int B) = 0$) верно
         $$\lim_{n\rightarrow+\infty}P(\xi_n\in B) = P(\xi \in B).$$

         \item Для любого борелевского регулярного множества $B$ верно
         $$\lim_{n\rightarrow+\infty} \EE \mathbbm{1}_B(\xi_n) = \EE \mathbbm{1}_B(\xi).$$

         \item Для любой ограниченной непрерывной на $\R$ функции $f$ верно
         $$\lim_{n\rightarrow+\infty} \EE f(\xi_n) = \EE f(\xi).$$

         \item $\phi_n\rightarrow \phi$ поточечно;
     \end{enumerate}
 \end{theorem}

 \begin{proof}
    $ $\medskip
    
     $3 \Ra 2$. Возьмём $ U = \mathbb{R}\setminus A$, тогда
    $P(\xi_n\in U ) = 1 - P(\xi_n\in A)$ и
    $$\varliminf_{n\rightarrow+\infty} P(\xi_n\in U ) = 1 - \varlimsup_{n\rightarrow+\infty} P(\xi_n\in A) \ge 1-P(\xi\in A) = P(\xi\in U ).$$
    
    $2 \Ra 3$. Аналогично предыдущему.

     $4 \Ra 5$. Понятно, что $P(\xi_n\in B) = \EE \mathbbm{1}_B(\xi_n)$ и $P(\xi \in B) = \EE \mathbbm{1}_B(\xi)$, так что
     $$\lim_{n\rightarrow+\infty}P(\xi_n\in B)= \lim_{n\rightarrow+\infty} \EE \mathbbm{1}_B(\xi_n) = \EE \mathbbm{1}_B(\xi)  = P(\xi \in B).$$

    $5 \Ra 4$. Аналогично предыдущему.
    
     $6 \Ra 7$. 
     $$\phi_\xi(t) = \EE e^{it\xi} = \EE (\cos(\xi t)) + i\EE (\sin(\xi t)).$$

     $2$ + $3 \Ra 4$. Пусть $A=\Cl B$, $ U  = \Int B$ $P(\xi\in A\setminus  U ) = 0$. Тогда
     $$\varlimsup P(\xi_n\in B) \le \varlimsup P(\xi_n\in A) \le P(\xi \in A) = P(\xi\in B) \le \varliminf P(\xi_n \in A) \le \varliminf P(\xi_n \in B).$$
    Значит, предел существует и он равен $P(\xi\in B)$.

     $1 \Ra 2$. Про открытое знаем, что
     \begin{gather*}
          U  = \bigsqcup\limits_{k=1} ^ {+\infty} (a_k, b_k]\fwhere P(\xi = a_k) = P(\xi = b_k) = 0.\\
         P(\xi_n\in (a_k, b_k]) = F_n(b_k) - F_n(a_k)\rightarrow F(b_k) - F(a_k) = P(\xi \in (a_k, b_k]);
     \end{gather*}
     Получаем, что
     \begin{align*}
         P(\xi_n\in U )& \ge P(\xi_n\in \bigsqcup\limits_{k=1} ^ m (a_k, b_k]) = \sum\limits_{k=1} ^ m P(\xi_n\in (a_k, b_k]) \\& \underset{m\rightarrow \infty}{\rightarrow}\sum\limits_{k=1} ^ m P(\xi\in(a_k, b_k]) = P(\xi\in \bigsqcup\limits_{k=1} ^ m(a_k, b_k]);
     \end{align*}
    Возьмём нижний предел:
     $$ \varliminf  P(\xi_n\in U ) \ge \varliminf  P(\xi_n\in \bigsqcup\limits_{k=1} ^ m  (a_k, b_k]) = \lim \ldots =
         P(\xi\in \bigsqcup\limits_{k=1} ^ m (a_k, b_k])\underset{m\rightarrow \infty}{\rightarrow} P(\xi\in U ).$$
     Значит, $\varliminf  P(\xi_n\in U ) \ge P(\xi\in U )$.

     $5\Ra 6$. Обозначим $D = \{x\in\mathbb{R}\mid P(f(\xi) = x) > 0\} = \{x\mid P_\xi(f^{-1}) > 0\}$ ---не более, чем счётное (такие точки --- точки разрыва функции распределения).
    По условию $f\in C(\mathbb{R})$ и $\abs{f}\le M$. Нарежем $[-M, M]$ на маленькие кусочки: $-M = t_0 < t_1 < \ldots < t_n = M$ такие, что $\forall i \ t_i \notin D$ (такое возможно, так как $D$ не более чем счетно).
     Введем следующие множества $A_j, B_j, U_j$:
     $$A_j = \{x\mid t_{j - 1}\le f(x) \le t_j\}\supset B_j = \{x\mid t_{j - 1}\le f(x) < t_j\} \supset \{x\mid t_{j - 1}< f(x) < t_j\} =  U _j.$$
     $A_j$ ~--- замкнуто, $ U _j$ ~--- открыто и $P(f(\xi)\in A_j\setminus  U _j) = 0$, потому что это означает, что $f(\xi)\in {t_{j - 1}, t_j}$, а эти точки не лежат в $D$, поэтому у них вероятность нулевая. Итого, $B_j$ ~--- регулярное
     Из этих $B_j$ теперь соорудим ступенчатую функцию 
     $$g(x)= \overset{m}{\underset{j = 1}{\sum}} t_j\cdot\mathbbm{1}_{B_j}(x).$$
    По 5-му пункту $\lim \EE g(\xi_n) = \EE g(\xi).$
    Заметим, что
       $$ \abs{f(x) - g(x)} = \max\{\abs{t_j - t_{j - 1}}\}, $$
       тогда это верно и для матожидания:
       $$\abs{\EE f(\xi) - \EE g(\xi)}\le \EE \abs{f(\xi) - g(\xi)}\le \max \abs{t_j - t_{j - 1}}.$$
        \begin{align*}
        \abs{\EE f(\xi_n) - \EE f(\xi)} &\le \abs{\EE f(\xi_n) - \EE g(\xi_n)} + \abs{\EE g(\xi_n) - \EE g(\xi)} + \abs{\EE g(\xi) - \EE f(\xi)} \\&< 2 \e + \abs{\EE g(\xi_n) - \EE g(\xi)} < 3 \e.
    \end{align*}
     Последнее верно при достаточно большом $n$. 
 \end{proof}\newpage
