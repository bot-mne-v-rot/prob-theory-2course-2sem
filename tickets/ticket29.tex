\section{Производящие функции для целозначных случайных величин. Примеры}

\begin{definition}
    Пусть $\xi \colon \Omega\rightarrow \mathbb{N}_0$ --- случайная величина. Тогда её \textit{производящей функцией} называется
    $$G_\xi(z)= \overset{\infty}{\underset{n = 0}{\sum}} P(\xi = n)z^n.$$
\end{definition}

\begin{properties}[производящей функции]
\enewline
    \begin{enumerate}
        \item $G_\xi(z) = \EE z^\xi$;

        \item $G_\xi(1) = 1$ и ряд сходится в единичном круге.

        \item $G_\xi'(1) = \EE \xi$;

        \item $\EE \xi^2 = G_\xi''(1) + G_\xi'(1)$;

        \item $\DD\xi = G_\xi''(1) + G_\xi'(1) - (G_\xi'(1))^2$;

        \item Если $\xi$ и $\eta$ независимы, то $G_{\xi + \eta}(z) = G_\xi(z)\cdot G_\eta(z)$.

           
    \end{enumerate}
\end{properties}
\begin{proof}
 \enewline
 \begin{enumerate}
     \item $\xi$ действует в неотрицательные числа, так что
     $$\EE z^\xi = \int_\mathbb{R} z^x \dd P_\xi(x) = \overset{\infty}{\underset{n = 0}{\sum}} P(\xi = n)z^n.$$
     \item Подставим $1$ в определение, получим сумму вероятностей всех возможных событий, что равняется единице. Коэффициенты --- некоторые вероятности, --- неотрицательны, так что в единичном круге есть сходимость.
     \item По определению
     \begin{gather*}
         G_\xi'(z) = \overset{\infty}{\underset{n = 1}{\sum}}nP(\xi = n)z^{n - 1};\\
         G_\xi'(z) = \overset{\infty}{\underset{n = 1}{\sum}}nP(\xi = n),
     \end{gather*}
     что в точности равно $\EE \xi$.
     \item По определению
      $$G_\xi''(z) = \overset{\infty}{\underset{n = 1}{\sum}}n(n - 1)P(\xi = n)z^{n - 2}$$
      Пользуясь предыдущим пунктом, получим
      $$G_\xi''(1) + G_\xi'(1) = \overset{\infty}{\underset{n = 1}{\sum}}n^2P(\xi = n),$$
      что в точности равно $\EE \xi^2$.
      \item По свойствам дисперсии $\DD \xi = \EE \xi^2 - (\EE \xi)^2$. Подставим два предыдущих пункта.
      \item Рассмотрим $G(z) = G_\xi(z)\cdot G_\eta(z)$. Это свёртка последовательностей. Коэффициент при $n$-ой степени равен  $c_n = P(\xi = 0)P(\eta = n) + P(\xi =1 )P(\eta = n - 1) + \ldots + P(\xi = n)P(\eta = 0) = P(\xi + \eta = n)$, а значит $G(z) = G_{\xi + \eta}(z)$. \qedhere
 \end{enumerate}
\end{proof}


    \begin{examples}

        \begin{enumerate}
            \item Равномерное распределение. Пусть $\xi$ равновероятно  принимает значения из $\{0, 1, \ldots n-1\}$. Тогда 
                  $$G_\xi = \frac{1+z+\ldots + z^{n-1}}{n} = \frac{1-z^n}{n(1-z)}.$$
                  Чтобы было удобнее работать, сделаем замену $z = 1 + t$:
                  $$G_\xi(1+t) = \frac{(1+t)^n - 1}{nt} = \frac{1}{n}\binom{n}{1} + \frac{1}{n}\binom{n}{2}t+\frac{1}{n}\binom{n}{3}t^2\ldots.$$
                  И тогда можем вычислить
                  \begin{gather*}
                      G_\xi'(1) = \frac{n- 1}{2};\\
                      G_\xi''(1) = \frac{(n - 1)(n-2)}{3};\\
                      \DD\xi = \frac{n^2 - 1}{12}.
                  \end{gather*}

            \item Задача Галилея. Бросается три кубика, какова вероятность что сумма очков равна 10?.
               Пусть $\xi_i$ --- количество очков на $i$-м кубике.
                  \begin{gather*}
                      G_{\xi_i}(z) = \frac{z+z^2+\ldots + z^6}{6} = \frac{z(1-z^6)}{6(1-z)};\\
                      G_{\xi_1 + \xi_2 + \xi_3}(z) = (G_\xi(z))^3 =  \frac{z^3(1-z^6)^3}{6^3(1-z)^3} = \frac{1}{6^3}z^3(1-3z^6 + 3z^{12} -z^{18})
                      \overset{\infty}{\underset{n = 0}{\sum}} \binom{n+2}{n}z^n;\\
                  \end{gather*}
                   По определению производящей функции коэффициент при $z_{10}$ --- это $P(\xi_1 + \xi_2 + \xi_3 = 10)$, как раз то, что мы ищем.
                  $$P(\xi_1 + \xi_2 + \xi_3 = 10) = \frac{1}{6^3} \left(\binom{9}{7} - \binom{3}{1}\right) = \frac{1}{8}.$$
        \end{enumerate}
    \end{examples}
