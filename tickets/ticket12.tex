\section{Дискретное, непрерывное и абсолютно непрерывное распределения. Свойства}

\begin{definition}
    Случайная величина $\xi$ --- \textit{дискретная}(или говорят, что она имеет \textit{дискретное распределение}), если
     $$\xi \colon \Omega \rightarrow Y,$$
     где множество $Y$ не более, чем счётно.
 \end{definition}
В дискретном вероятностном пространстве распределение устроено следующим образом:
  $$P_\xi(A) = \underset{k: y_k\in A}{\sum} P(\xi = y_k).$$
  Таким образом, распределение полностью определяется вероятностями $P(\xi = y_k)$.

 \begin{definition}Случайная величина имеет \textit{непрерывное распределение}, если её функция распределения непрерывна.
 \end{definition}
Как мы уже знаем, это равносильно тому, что  $\forall x\in\mathbb{R}$  $P(\xi = x) = 0$.
 \begin{definition}Случайная величина $\xi$ имеет \textit{абсолютно непрерывное распределение}, если у её функции распределения есть плотность относительно меры Лебега, то есть $p_\xi (t) \ge 0$ измеримая по Лебегу такая, что
     $$P_\xi(A) = \int_A p_\xi(t)\dd t.$$
 \end{definition}
 
Понятно, что для функция распределения такой величины считается как 
     $$F_\xi (x) = \int_{-\infty}^x p_\xi (t) \dd t.$$

 \begin{properties}[плотности распределения]
 \enewline
     \begin{enumerate}
         \item $P(\xi\in A) = P_\xi(A) = \intt_A p_\xi (t) \dd t$;
         \item $F_\xi(x) = \intt_{-\infty}^x p_\xi(t) \dd t$;
         \item Если $t_0$ --- точка непрерывности $p_\xi$, то
               $p_\xi(t_0) = F_\xi'(t_0)$ (на самом деле равенство есть почти везде, так как монотонно возрастающая функция дифференцируема почти везде);
     \end{enumerate}
 \end{properties}
