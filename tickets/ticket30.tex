\section{Математическое ожидание для комплекснозначных случайных величин. Свойства. Ковариация}

Определим комплекснозначные случайные величины.

 \begin{definition}
     $\xi\colon \Omega\rightarrow\mathbb{C}$ --- \textit{случайная величина}, если $\Re\xi$ и $\Im\xi$~--- вещественнозначные случайные величины;
     $\xi = \Re\xi + i\Im\xi$, $\EE \xi = \EE (\Re\xi) + i\EE (\Im\xi)$.
 \end{definition}

 \begin{properties}[комплекснозначной случайной величины]
\enewline
     \begin{enumerate}
         \item Комплексная линейность;
                

         \item $\abs{\EE \xi} \le \EE \abs{\xi}$;

               
     \end{enumerate}
 \end{properties}

\begin{proof}
 \enewline
     \begin{enumerate}
         \item Докажем, что $\EE (i\xi) = i\EE \xi$. Пусть $\xi = \zeta + i\eta$, где $\zeta, \eta\colon \Omega\rightarrow\mathbb{R}$. Тогда
               $$\EE (i\xi) = \EE (i\zeta - \eta) = i\EE \zeta - \EE \eta = i(\EE \zeta + i\EE \eta) = i\EE \xi.$$
               \item Возьмем $c=\frac{\overline{\EE \xi}}{\abs{\EE \xi}}$, тогда $\abs{c} = 1$, $\EE (c\xi) = \abs{\EE \xi}$.
               $$\EE \abs{\xi} = \EE \abs{c\xi} \ge \EE \abs{\Re(c\xi)} \ge \EE (\Re(c\xi)) = \Re \EE (c\xi) = \abs{\EE \xi}.$$
               \qedhere
     \end{enumerate}
\end{proof}
 \begin{definition} \textit{Ковариацией комплекснозначных} случайных величин $\xi$ и $\eta$ называется
     $$\cov (\xi, \eta) = \EE ((\xi -\EE \xi)\overline{(\eta - \EE \eta)}).$$
 \end{definition}
 
 Понятно, что для величин, принимающих вещественное значение определение не изменилось.
 
 Также заметим, что сохранилось равенство
 $$\DD\xi = \cov (\xi, \xi).$$