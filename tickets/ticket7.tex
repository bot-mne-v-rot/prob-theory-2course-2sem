\section{Локальная теорема Муавра–Лапласа. Пример}


\begin{theorem}[Локальная предельная теорема Муавра-Лапласа]
    Пусть $0 < p < 1,\; q = 1 - p$. $T$ --- некоторое число. Обозначим $x = \frac{k - np}{\sqrt{npq}}$, причём $k$ меняется так, что $\abs{x}\le T$ при $n \rightarrow +\infty$.
    Тогда 
    $$P(S_n = k) \underset{n\rightarrow +\infty}{\sim} \frac{e^{-\frac{x^2}{2}}}{\sqrt{2\pi npq}} \ \text{равномерно по $k$}.$$
\end{theorem}

\begin{proof}
   Пусть $n \to \infty$. Из условий
    $$np + T\sqrt{npq} \ge k = np + x\sqrt{npq} \ge np - T\sqrt{npq}.$$
    Тогда $ k\rightarrow +\infty$ верно из 2-го неравенства и $n - k \rightarrow +\infty$ из 1-го (если из $n$ вычесть $np + T\sqrt{npq}$,
    то это будет стремится к $+\infty$).
    Обозначим
    $$\alpha = \frac{k}{n} = p + x\sqrt{\frac{pq}{n}} \rightarrow p \fand \beta = \frac{n - k}{n} = 1 - \alpha = q - x\sqrt{\frac{pq}{n}} \rightarrow q.$$
   Тогда
   \begin{align*}
       P(S_n = k) &= \binom{n}{k}p^kq^{n- k} \sim \frac{n^ne^{-n}\sqrt{2\pi n}p^kq^{n - k}}{k^ke^{-k}\sqrt{2\pi k}(n - k)^{n - k}e^{k - n}\sqrt{2\pi(n - k)}} \\&=
        \frac{p^kq^{n - k}}{(\frac{k}{n})^k (1 - \frac{k}{n})^{n - k}\sqrt{2\pi n}\sqrt{\frac{k}{n}(1 - \frac{k}{n})}} \\&\sim
        \frac{p^kq^{n - k}}{\alpha^k\beta^{n - k}\sqrt{2\pi npq}}.
   \end{align*}
   Надо доказать, что
    $$\frac{p^kq^{n - k}}{\alpha^k\beta^{n - k}} \overset{}{\sim} e^{-\frac{x^2}{2}},$$
что равносильно тому, что
    \begin{gather*}
    k\ln\left(\frac{\alpha}{p}\right) + (n - k)\ln\left(\frac{\beta}{q}\right) \overset{}{\sim} \frac{x^2}{2}.\\
       \frac{\alpha}{p} = 1 + x\sqrt{\frac{q}{np}}, \  \ln\left(\frac{\alpha}{p}\right) = x\sqrt{\frac{q}{np}} - x^2 \frac{q}{2np}
        + O\left(\frac{1}{n^\frac{3}{2}}\right).\\
        \frac{\beta}{q} = 1 - x\sqrt{\frac{pq}{n}}, \  \ln\left(\frac{\beta}{q}\right) = -x\sqrt{\frac{p}{nq}} - x^2 \frac{p}{2nq}
        + O\left(\frac{1}{n^\frac{3}{2}}\right).
   \end{gather*}
    Итого, подставив, получаем
   \begin{align*}
       k\ln\left(\frac{\alpha}{p}\right) + (n - k)\ln\left(\frac{\beta}{q}\right) &= (np + x\sqrt{npq})\left(x\sqrt{\frac{q}{np}} - x^2 \frac{q}{2np}
        + O\left(\frac{1}{n^\frac{3}{2}}\right)\right) \\& \quad\quad + (nq - \sqrt{npq})\left(-x\sqrt{\frac{p}{nq}} - x^2 \frac{p}{2nq}
        + O\left(\frac{1}{n^\frac{3}{2}}\right)\right) \\&= x\sqrt{npq} + x^2q - x^2\frac{q}{2} + O\left(\frac{1}{\sqrt{n}}\right) \\& \quad\quad - x\sqrt{npq}
        +x^2p-x^2\frac{p}{2} + O\left(\frac{1}{\sqrt{n}}\right) \\&= \frac{x^2}{2} + O\left(\frac{1}{\sqrt{n}}\right).
   \end{align*}
  

\end{proof}

\begin{example}На рулетке 37 секторов --- 18 красных и 18 черных и 1 зелёный. Игрок участвует в $n = 222$ раундах. Посчитаем шанс ``отбить"":
    $$P(S_{222} = 111) = \binom{222}{111}(\frac{18}{37})^{111}(\frac{19}{37})^{111} \approx 0,0493228\ldots.$$
    Теорема Муавра-Лапласа дает нам оценку
   $$ P(S_{222} = 111)\approx \frac{e^{-\frac{(k - np)^2}{2npq}}}{\sqrt{2\pi npq}}\approx 0,0493950\ldots.$$
\end{example}
