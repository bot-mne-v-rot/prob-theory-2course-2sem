\section{Вероятностное пространство. Условная вероятность. Независимые события}

\begin{definition} \textit{Вероятностное пространство} --- это $(\Omega, \mathcal{F}, P),$ где
    $\Omega$ --- множество элементарных событий;
    $\mathcal{F}$ ---  $\sigma$-алгебра подмножеств $\Omega$, его элементы --- \textit{случайные события};
    $P$ --- вероятностная мера на $\mathcal{F}$, такая что $P(\Omega) = 1$.
\end{definition}


\begin{definition} \textit{Условная вероятность}.
    Пусть $B \in \mathcal{F}, P(B) > 0$. Тогда
    $P(A|B) = \frac{P(A\cap B)}{P(B)}$~--- вероятность $A$ при условии $B$.

\end{definition}

\begin{definition}
    События $A$ и $B$ независимы, если 
    $P(A \cap B) = P(A) \cdot P(B)$.

\end{definition}

\begin{definition}
    Множество событий $A_i$ по $i \in I$ является независимым в совокупности, если
    $\forall i_1, ..., i_k \in I$

    $P(A_{i_1}\cap A_{i_2} \cap\ldots\cap A_{i_k}) = P(A_{i_1})\cdot P(A_{i_2})\cdot\ldots\cdot P(A_{i_k})$.

\end{definition}