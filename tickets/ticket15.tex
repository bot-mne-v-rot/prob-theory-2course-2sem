\section{Совместная функция распределения и совместная плотность. Функции распределения и плотности для независимых случайных величин}


\begin{definition} \textit{Совместной функцией распределения} вектора величин $\vv{\xi} \colon \Omega \rightarrow \mathbb{R}^n$ называется
    $$F_{\vv{\xi}}(\vv{x}) = P(\xi_1 \le x_1, \ldots, \xi_n \le x_n).$$
\end{definition}

\begin{definition}\textit{Совместная плотность распределения} вектора величин $\vv{\xi}$ --- это такая функция $p_{\vv{\xi}}(\vv{t})$, измеримая в $\mathbb{R}^n$ (если она существует), что
    $$P_{\vv{\xi}}(A) =\int\limits_A p_{\vv{\xi}}(\vv{t})\dd\lambda_n(\vv{t}).$$
\end{definition}

Как и в случае с одномерной плотностью, нетрудно понять, что
    $$F_{\vv{\xi}}(\vv{x}) =
        \int\limits_{-\infty}^{x_1}\ldots\int\limits_{-\infty}^{x_n}p_{\vv{\xi}}(\vv{t})\dd t_n\ldots \dd t_1.$$

\begin{corollary}
    \begin{enumerate}
        \item $\xi_1, \ldots, \xi_n$ --- независимые $\iff$ $F_{\vv{\xi}}(\vv{x}) = F_{\xi_1}(x_1)\ldots F_{\xi_n}(x_n)$.

         

        \item Пусть $\xi_1, \ldots, \xi_n$ --- абсолютно непрерывные случайные величины.
              Тогда $\xi_1, \ldots, \xi_n$ независимы тогда и только тогда, когда $p_{\vv{\xi}}(\vv{t}) = p_{\xi_1}(t_1)\cdot\ldots\ldots p_{\xi_n}(t_n)$.

     
    \end{enumerate}
\end{corollary}

\begin{proof}
\enewline
\begin{enumerate}
      
                     \item[2.]   $\Longleftarrow$. По определению
                     $$P_{\vv{\xi}} (A_1\times\ldots\times A_n) = \int\limits_{A_1\times\ldots\times A_n} p_{\vv{\xi}}(\vv{t}) \dd\lambda_n(t) = \int\limits_{A_1\times\ldots\times A_n} p_{\xi_1}(t_1)\dd t_1\cdot\ldots\cdot p_{\xi_n}(t_n)\dd t_n.$$
                  По теореме Фубини-Тонелли это равно произведению интегралов:
                  \begin{align*}
                      \int\limits_{A_1\times\ldots\times A_n} p_{\xi_1}(t_1)\dd t_1\cdot\ldots\cdot p_{\xi_n}(t_n)\dd t_n &= \int\limits_{A_1} p_{\xi_1}(t_1)\dd t_1 \cdot\ldots\cdot \int\limits_{A_n} p_{\xi_n}(t_n)\dd t_n \\&= P_{\xi_1}(A_1)\cdot\ldots\cdot P_{\xi_n}(A_n).
                  \end{align*}
                  

                  $\Ra$. Проверим, что
                  $p_{\xi_1}(t_1)\cdot\ldots\cdot p_{\xi_n}(t_n)$ --- совместная плотность.
                  Для этого докажем равенство на ячейках, то есть проверим равенство
                  $$P_{\vv{\xi}} (a, b] =
                      \underset{(a, b]}{\int} p_{\xi_1}(t_1)\cdot\ldots\cdot p_{\xi_n}(t_n) \dd\lambda_n.$$
                   С одной стороны,
                   $$P_{\vv{\xi}} (a, b] = \overset{n}{\underset{k = 1}{\prod}} P_{\xi_k} (a_k, b_k].$$
                   С другой, по теореме Тонелли,
                   $$\underset{(a, b]}{\int} p_{\xi_1}(t_1)\cdot\ldots\cdot p_{\xi_n}(t_n) \dd\lambda_n =
                      \overset{n}{\underset{k = 1}{\prod}}  \underset{(a_k, b_k]}{\int} p_{\xi_k}(t_k) \dd t_k.$$
                   Осталось показать, что $\intt_{(a_k, b_k]}p_{\xi_k}(t_k) \dd t_k = P_{\xi_k} (a_k, b_k]$, ну а это является определением плотности. \qedhere
\end{enumerate}
\end{proof}