\section{Ковариация. Связь с независимостью. Коэффициент корреляции. Моменты случайной величины}

\begin{definition}
    \textit{Ковариацией} случайных величин $\xi$ и $\eta$ называется
     $$\cov(\xi, \eta) = \EE ((\xi - \EE \xi)(\eta - \EE \eta)).$$
 \end{definition}

 \begin{properties}[ковариации]
\enewline
     \begin{enumerate}

         \item $\cov(\xi, \xi) = \DD \xi$.


         \item $\cov(\xi, \eta) = \EE (\xi\eta) - \EE \xi \EE \eta$.


         \item $\DD (\xi + \eta) = \DD \xi + \DD \eta + 2\cov(\xi, \eta)$.

         \item $\DD (\overset{n}{\underset{k = 1}{\sum}}\xi_k) = \overset{n}{\underset{k = 1}{\sum}}\DD \xi_k + \underset{i\neq k}{\sum}\cov(\xi_i, \xi_k) =
                   \overset{n}{\underset{k = 1}{\sum}}\DD \xi_k + 2\underset{i < k}{\sum}\cov(\xi_i, \xi_k)$.

         \item Если $\xi$ и $\eta$ независимые, то $\cov(\xi, \eta) = 0$.

         \item $\cov(\xi, \eta) = \cov(\eta, \xi)$.


         \item $\cov(c\xi, \eta) = c\cdot \cov(\xi, \eta)$.


         \item $\cov(\xi_1+\xi_2, \eta) = \cov(\xi_1, \eta) + \cov(\xi_2, \eta)$.
     \end{enumerate}
 \end{properties}

Сделаем несколько важных замечаний.

     \begin{enumerate}
         \item Дисперсия и ковариация могут не существовать (например, если не определено матожидание). Для существования ковариации надо, чтобы $\EE \xi^2 < +\infty$ и $\EE \eta^2 < +\infty$ (а дисперсия --- частный случай ковариации).

         \item Из $\cov(\xi, \eta) = 0$ не следует независимость этих величин.

               Например, пусть $\Omega = \{0, \frac{\pi}{2}, \pi\}$, каждое значение равновероятно. Возьмём
               $\xi = \sin\omega$, $\eta = \cos\omega$. Тогда 
               \begin{gather*}
                   \xi\eta\equiv 0, \ \EE (\xi\eta) = 0, \ \EE \eta = 0;\\
                   \cov(\xi, \eta) = \EE (\xi\eta) - \EE \xi \EE \eta = 0.
               \end{gather*}
               Но они не являются независимыми:
               $$P(\xi = 1, \eta = 1) = 0 \neq P(\xi = 1)P(\xi = 1) = \frac{1}{9}.$$
     \end{enumerate}

 \begin{definition}\textit{Коэффициентом корреляции} случайных величин $\xi$ и $\eta$ называется
     $$\rho(\xi, \eta) = \frac{\cov(\xi, \eta)}{\sqrt{\DD \xi}\sqrt{\DD \eta}}.$$
 \end{definition}
Очевидно, это значение лежит на отрезке $[-1, 1]$.
 \begin{definition}
     Случайные величины, для которых $\cov(\xi, \eta) = 0$ называются \textit{некоррелированными}.
 \end{definition}

 \begin{definition}
    $\EE \xi^k = \int\limits_\mathbb{R}x^k \dd P_\xi(x)$ --- \textit{$k$-й момент} случайной величины. 
    $\EE \abs{\xi}^k$--- \textit{$k$-й абсолютный момент}. 
    $\EE ((\xi - \EE \xi)^k)$ --- \textit{$k$-й центральный момент}.
\end{definition}

Легко заметить, что дисперсия --- 2-й центральный момент.
\newpage
