\section{Интегральная теорема Муавра–Лапласа и оценка на скорость сходимости (без доказательства). Неулучшаемость показателя степени в оценке. Задача о театре. Случайное блуждание на прямой}

\begin{theorem}[Интегральная предельная теорема Муавра-Лапласа]\label{th:ipt}
    Пусть $0<p<1$.
    Тогда $$P(a< \frac{S_n - np}{\sqrt{npq}}\le b) \underset{n\rightarrow\infty}{\rightarrow} \frac{1}{\sqrt{2\pi}} \int_a^b e^{-\frac{x^2}{2}}\dd x \ \text{ равномерно по } a, b\in \mathbb{R}.$$
\end{theorem}

\begin{notation}
$$\Phi(x) = \frac{1}{\sqrt{2\pi}} \int_{-\infty}^x e^{-\frac{t^2}{2}}\dd t.$$
$$\Phi_0(x) = \frac{1}{\sqrt{2\pi}} \int_{0}^x e^{-\frac{t^2}{2}}\dd t.$$
\end{notation}

Эти функции не выражаются через элементарные. Из матанализа знаем значения в некоторых точках. Например про второй знаем, что $\Phi_0(x) \approx \frac{1}{2}$ при $x > 4$. 

\begin{theorem}[Оценка скорости сходимости. Частный случай теоремы Берри-Эссена]
    $$\underset{x\in\mathbb{R}}{\sup}\abs{P( \frac{S_n - np}{\sqrt{npq}} \le x) - \Phi(x)} \le
        \frac{1}{2}\frac{p^2 + q^2}{\sqrt{npq}}.$$
\end{theorem}

\begin{example} [Неулучшаемость оценки]
   Приведем пример, показывающий, что сходимоcть не может быть быстрее, чем $\frac{C}{\sqrt{n}}$.
   
   Пусть $p=q=\frac{1}{2}, x = 0$. Точное значение $\Phi(x)$ мы знаем: $\Phi(0) = \frac{1}{2}$, так как это половина всего распределения. Теперь оценим вероятность: \[P\left(\frac{S_n - \frac{n}{2}}{\sqrt{n\frac{1}{4}}} \le 0\right) = P\left(S_n \le \frac{n}{2}\right) =  P(S_{2n} \le n) = \]
   Заметим, что $P(S_{2n} \le n) = P(S_{2n} \ge n)$,  причем объединение этих двух событий дает все вероятностное пространство за исключением того, что событие $P(S_{2n} = n)$ посчитано 2 раза. Отсюда вытекает равенство: \[ = \frac{1}{2} + \frac{P(S_{2n} = n)}{2} = \frac{1}{2} + \frac{1}{2} \cdot \binom{2n}{n}\cdot\frac{1}{4^n} \]
   Осталось вспомнить, что $\binom{2n}{n}\cdot\frac{1}{4^n} \sim \frac{1}{\sqrt{\pi n}}$, поэтому: \[\left|P\left(\frac{S_n - \frac{n}{2}}{\sqrt{n\frac{1}{4}}} \le 0\right) - \Phi(0)\right| \sim \frac{1}{\sqrt{\pi n}} \]
\end{example}

\begin{example}[задача о театре]
    В театре, рассчитанном на $n = 1600$ мест, есть два гардероба.
    Сколько должно быть мест в каждом гардеробе, чтобы в среднем не чаще раза в месяц какому-то посетителю пришлось идти не к ближайшему из-за отсутствия в нем мест(считаем, что люди заходят в гардеробы равновероятно)?
    Обозначим число вешалок в каждом из гардеробов за $C$.
    Пусть $S_n$ ~--- число людей, сдавших вещи в первый гардероб. Тогда должны выполняться неравенства $S_n \le C$ и $n - S_n \le C$.
    Хотим, чтобы выполнялось
    $$P(n - C \le S_n \le C) \approx \frac{29}{30}.$$
   По теореме \ref{th:ipt}
   \begin{align*}
       P(n - S_n \le S_n \le C) &= P(\frac{-C+\frac{n}{2}}{\sqrt{npq}} \le \frac{S_n-np}{\sqrt{npq}} \le \frac{C-\frac{n}{2}}{\sqrt{npq}}) \\&= P(\frac{-C + 800}{20} \le \frac{S_n - np}{\sqrt{npq}}\le \frac{C - 800}{20}) \approx \frac{1}{\sqrt{2\pi}}\int_{-\frac{C - 800}{20}}^{\frac{C - 800}{20}}e^{-\frac{t^2}{2}}\dd t \\&=
        2\Phi_0(\frac{C-800}{20})\approx \frac{29}{30} \Ra C\approx 843.
   \end{align*}
    
\end{example}

\begin{example} [Случайное блуждание на прямой]
    Ходим по прямой, начиная с 0. Идем на один шаг вправо с вероятностью $p$, и на один шаг влево с вероятностью $q=(1-p)$. Заметим, что точка, в которую мы придем, выражается как $a_n = 2S_n - n$. Тогда вероятность придти в конкретную точку после $n$ шагов вычисляется как: $P(a_n = k) = P(S_n = \frac{k+n}{2}) = \binom{n}{\frac{n+k}{2}}p^{\frac{n+k}{2}}q^{\frac{n-k}{2}}$, при условии, что $n$ и $k$ одной четности.
\end{example}
\newpage
