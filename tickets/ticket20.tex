\section{Дисперсия. Свойства дисперсии. Неравенство Чебышёва. Математическое ожидание и дисперсия для равномерного и нормального распределений}


\begin{definition}\textit{Дисперсией} случайной величины $\xi$ называется
    $$\DD \xi = \EE (\xi - \EE \xi)^2.$$
\end{definition}

\begin{properties}[дисперсии]
\enewline
    \begin{enumerate}
        \item $\DD \xi\ge0$ и если $\DD \xi = 0$, то $P(\xi = c) = 1$, где $c$ --- некоторая константа. 
        \item $\DD \xi = \EE \xi^2 - (\EE \xi)^2$.

              
        \item $\DD (c \cdot \xi) = c^2\DD \xi$ (в частности, $\DD \xi = \DD (-\xi)$).


        \item Если $\xi$ и $\eta$ независимы, то $\DD (\xi+\eta) = \DD \xi+\DD \eta$.


        \item $\EE \abs{\xi - \EE \xi}\le \sqrt{\DD \xi}$.


        \item Неравенство Чебышёва при $t > 0$.
              $$P(\abs{\xi - \EE \xi} \ge t)\le \frac{\DD \xi}{t^2}.$$
              

    \end{enumerate}
\end{properties}

\begin{proof}
\begin{enumerate}
   \item Очевидно из определения;
   \item По определению
   $$\DD \xi = \EE (\xi - \EE \xi)^2 = \EE (\xi^2 - 2\xi \EE \xi + (\EE \xi)^2) = \EE \xi^2 - 2(\EE \xi)^2 + (\EE \xi)^2 = \EE \xi^2 - (\EE \xi)^2.$$

   \item      Очевидно из определения;            
            

   \item Воспользуемся свойством матожидания для произведения независимых величин:
   \begin{align*}
       \DD (\xi+\eta) &= \EE (\xi +\eta)^2 - (\EE (\xi +\eta))^2 \\&= \EE \xi^2 + 2\EE \xi\eta + \EE \eta^2 - (\EE \xi + \EE \eta)^2 \\&= \EE \xi^2 + \EE \eta^2 - (\EE \xi)^2 - (\EE \eta)^2.
   \end{align*}
                  
           
              
   
           
       \item Пусть $\eta =\xi - \EE \xi$, тогда по неравенству Ляпунова
       $$\EE \abs{\eta} \le (\EE \abs{\eta}^2)^\frac{1}{2}.$$
             
             
       \item Из предыдущего свойства
                   $$P(\eta\ge t) \le \frac{\EE \eta^2}{t^2}.$$ \qedhere
                   \end{enumerate}
              \end{proof}
\begin{examples}

    \begin{enumerate}
        \item Пусть $\xi\sim\mathcal{U}[0, 1]$. Тогда
        \begin{gather*}
            \EE \xi = \int\limits_\mathbb{R}x\dd P_\xi = \int\limits_0^1x\dd x = \frac{x^2}{2}\Bigg|_0^1  = \frac{1}{2}.\\
            \EE \xi^2 = \int\limits_\mathbb{R}x^2\dd P_\xi(x) = \int\limits_0^1 x^2\dd x = \frac{x^3}{3}\Bigg|_0^1 = \frac{1}{3}.\\
            \DD \xi = \EE \xi^2 - (\EE \xi)^2 = \frac{1}{12}.
        \end{gather*}


        \item Пусть $\xi\sim\mathcal{U}[a, b]$. Нетрудно заметить, что $\xi = (b-a)\eta + a$, где $\eta\sim\mathcal{U}[0, 1]$. Тогда
        \begin{gather*}
            \EE \xi = \EE ((b-a)\eta + a) = (b-a)\EE \eta + a = \frac{a+b}{2}.\\
            \DD \xi = \DD ((b-a)\eta + a) = (b-a)^2\DD \eta = \frac{(b-a)^2}{12}.
        \end{gather*}


        \item Пусть $\xi\sim\mathcal{N}(0, 1)$. Тогда
              $$\EE \xi = \int\limits_\mathbb{R}x\dd P_\xi(x) = \int\limits_\mathbb{R} x\frac{1}{\sqrt{2\pi}}e^{-\frac{x^2}{2}}\dd x = 0,$$ поскольку функция нечётная.
              $$\DD \xi = \EE \xi^2 = \frac{1}{\sqrt{2\pi}}\int\limits_\mathbb{R}x^2e^{-\frac{x^2}{2}}\dd x = -\frac{1}{\sqrt{2\pi}}\int\limits_\mathbb{R}x\dd\left(e^{-\frac{x^2}{2}}\right)= \frac{1}{\sqrt{2\pi}} \int\limits_\mathbb{R} e^{-\frac{x^2}{2}}\dd x = 1.$$ 

        \item Пусть $\xi\sim\mathcal{N}(a, \sigma^2)$. Поймём, что
              $\xi = \sigma\eta + a$, где $\eta \sim \mathcal{N}(0, 1)$.
             Пусть $\xi' = \sigma\eta$. Тогда
             \begin{align*}
                 F_{\xi'}(x) &= \int\limits_{-\infty}^x\frac{1}{\sqrt{2\pi}\sigma}e^{-\frac{t^2}{2\sigma^2}}\dd t
                   = [t = \sigma s] \\&= \frac{1}{\sqrt{2\pi}}\int\limits_{-\infty}^{\frac{x}{\sigma}}e^{-\frac{s^2}{2}}\dd s =
                  F_\eta\left(\frac{x}{\sigma}\right),
             \end{align*}
              то есть $$\xi' = \sigma\eta \sim \mathcal{N}(0, \sigma^2).$$
           Аналогично доказывается вторая часть, и тогда
           \begin{gather*}
               \EE \xi = \EE (\sigma\eta +a) = \sigma \EE \eta + a = a.\\
               \DD \xi = \DD (\sigma\eta + a) = \sigma^2.
           \end{gather*}
          
    \end{enumerate}
\end{examples}